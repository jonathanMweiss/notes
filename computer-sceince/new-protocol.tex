%% LyX 2.3.3 created this file.  For more info, see http://www.lyx.org/.
%% Do not edit unless you really know what you are doing.
\documentclass[oneside,english]{amsart}
\usepackage[T1]{fontenc}
\usepackage[latin9]{inputenc}
\usepackage{amstext}
\usepackage{amsthm}
\PassOptionsToPackage{normalem}{ulem}
\usepackage{ulem}

\makeatletter
%%%%%%%%%%%%%%%%%%%%%%%%%%%%%% Textclass specific LaTeX commands.
\numberwithin{equation}{section}
\numberwithin{figure}{section}
\theoremstyle{plain}
\newtheorem{thm}{\protect\theoremname}
\theoremstyle{definition}
\newtheorem{defn}[thm]{\protect\definitionname}
\theoremstyle{remark}
\newtheorem{claim}[thm]{\protect\claimname}

\makeatother

\usepackage{babel}
\providecommand{\claimname}{Claim}
\providecommand{\definitionname}{Definition}
\providecommand{\theoremname}{Theorem}

\begin{document}
Each user has $pk_{i}$, the full public key would be $\prod_{i}pk_{i}=pk$.
that is $\prod_{i}g^{x_{i}}=g^{\sum_{i}x_{i}}$ .

to add a new transaction $tx$: sample new keys: $\left(sk',pk'\right)$
and a random number $r$.
\begin{itemize}
\item compute $\gamma=Enc_{pk'}\left(tx\right)$
\item compute $c=\left(\underbrace{g^{r}}_{c_{L}},\underbrace{e\left(pk,H\left(\gamma\right)^{r}\right)\cdot sk'}_{c_{R}}\right)$ 
\item send $\left(c,\gamma,pk'\right)$.
\end{itemize}
Now, when every server signs: $\sigma_{i}=H\left(\gamma\right)^{sk_{i}}$\\
we can compute the product signature: $\sigma=\prod H\left(\gamma\right)^{sk_{i}}$
which can be used to decrypt the temp secret key: 
\begin{align*}
 & \frac{c_{R}}{e\left(c_{L},\sigma\right)}=\frac{e\left(g^{x},H\left(\gamma\right)^{r}\right)sk'}{e\left(g^{r},\sigma\right)}=\\
 & =\frac{e\left(g^{\sum sk_{i}},H\left(\gamma\right)^{r}\right)\cdot sk'}{e\left(g^{r},H\left(\gamma\right)^{\sum sk_{i}}\right)}=\\
 & =\frac{e\left(g^{\sum sk_{i}},H\left(\gamma\right)^{r}\right)\cdot sk'}{e\left(g,H\left(\gamma\right)\right)^{r\cdot\sum sk_{i}}}=\\
 & =\frac{e\left(g,H\left(\gamma\right)\right)^{r\sum sk_{i}}\cdot sk'}{e\left(g,H\left(\gamma\right)\right)^{r\cdot\sum sk_{i}}}=\\
 & =sk'
\end{align*}
 we managed to find out the temp secret key. now we can get the value
under $\gamma=Enc_{pk'}\left(tx\right)$. great :)


\subsection{Issues}
\begin{itemize}
  \item Can we verify the votes? otherwise we cannot catch a bad vote: $\sigma_i = H(\gamma)^{garbage}$
\end{itemize}

\section{New recovery protocol}
Assume in this protocol that \emph{proof} is trusted and conveniences that some node $i$ is faulty and
 that the proof contains the index/id of the faulty node.
Each node starts the protocol with an empty set of all proofs it had seen $A$.
\begin{itemize}
\item Upon a new proof:
  \begin{itemize}
    \item If the proof is valid, insert the new proof into $A$.

    \item If $|A| >f$ halt.

    \item Encrypt the shard $Enc_\text{scehme}(\text{shard}_i)$ with the cryptographic scheme and publish it along
      with the updated set $A$: $\text{\{A},Enc_\text{scehme}(\text{shard}_i)\}=msg$.
  \end{itemize}

\item A server that receives a new $msg$ will:
  \begin{itemize}
    \item Verify the new proof and run the previous step.
    \item if it has matching proof set $A$, sign to reveal the decrypted item in the message.
  \end{itemize}

\item decrypting a secret requires at least $2f+1$ signatures/votes.
\end{itemize}

This protocol ensures that every honest node will eventually hold a matching proof set.
In case of too many faults, the secrets remain hidden (needs a proof of correctness ?).

\subsection{analysis of time complexity:}
I assume we publish all keys at once, hidden with Gil's scheme.
We'll need $2f+1$ publications to decrypt Gil's scheme.
and $2f+1$ votes on each publication. Meaning $O(n^2)$ communications.

\subsection{The diff from BA?}
The protocol above does not conform to the Agreement definition.
\begin{defn}
Agreement
\begin{itemize}
\item No two honest nodes decide on a different value.
\item If all honest have the same input $v$, then $v$ must be the decision
value.
\item (termination) All honest nodes must eventually decide on a value in
$V$ (set of all valid values) and terminate.
\end{itemize}
\end{defn}


\subsection{Protocol and comparison:}

  Denote: $v\in V$ containing all indexes of faulty nodes, where $V=\mathcal{P}\left(\left[n\right]\right)$
  the power set of $\left[n\right]$. 

\begin{defn}
    A protocol that solves Mixnet Recovery is a protocol that given a system with multiple failures where each honest node $i$ holds some
    $v_{i}\in V$ must have the following properties:
  \begin{itemize}
    \item (safety) if $\left|\bigcup_{i}v_{i}\right|>f$, at most $f-1$ keys
    are released. (this property ensures that no matter what the nodes choose to release, we do not
     publish more than $f$ ).

    \item (availability) if $\left|\bigcup_{i}v_{i}\right|<f$, each honest
    node release a secret part of its shard for $key_{j}$ where $j\in v$
    (this ensures that we can release all keys from $\bigcup v_{i}$).

  \end{itemize}

    Notice that we only look at the values of the honest nodes at the
    end of the protocol. That is, after all messages were passed, $v_{i}$
    represents the indexes that node $i$ wishes to recover.
\end{defn}
    

\begin{claim}
  The system can remain safe even if the protocol isn't an Agreement protocol.
\end{claim}

\begin{proof}
We show that it is possible that two honest nodes \emph{decided} on two different values $v_{A},v_{B}$ and show that the
safety property still holds.
\begin{itemize}
  % \item from not agreement \exist two nodes without the same $v$, it might b2 2 but there might be more.
\item In this proof, we assume the protocol terminates. Otherwise, this is not interesting.
\item Without the loss of generality $v\in V$ represents the subset of
indexes a node will contribute its secret shares to construct a $key_{i}$
\item We use the minimal nodes such that $n\ge3f+1$.
\item We assume that $\left|v_{A}\cup v_{B}\right|>f$. Otherwise, the system
  is safe by default.
\end{itemize}

The protocol ensures that the set $v_i$ each node votes for monotonically increases in size throughout time. 
Let us describe the problem in terms of graphs:
We have $|\{ v| v\in V \land |v| < f \}|$ colors, $n$ vertices, which we divide into two types:
 $f$ Unique vertices can belong to any cluster at once, and Regular vertices can belong to one color, thus, one cluster.
Can we have two different clusters in our graph have at least $2f+1$ vertices?
No: We have $n=3f+1$ vertices and require clusters of $2f+1$ vertices. any two clusters have at least $f+1$ shared vertices.
So, there exist $f+1$ vertices that should be of at least two colors,
but we can only have $f$ such unique vertices.

The system is safe, although two or more nodes might exist without the same value at the end of the protocol.
\end{proof}


This proof ensures that even without the protocol being an Agreement
protocol, it is still safe.


\end{document}
