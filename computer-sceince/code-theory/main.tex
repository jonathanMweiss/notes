\chapter{Code Theory}

small sum about code theory. needed for a project of mine.

\section{Introduction}
We begin with some basic notions. What is a codeword? what is a code? what are symbols?
\begin{definition}
    Let $A=\{a_1, a_2, \dots a_q \}$ be a set of size $q$.
    We refer to $A$ as the \emph{alphabet} of the code. and each element is a \emph{code symbol}.

    \begin{enumerate}
        \item A q-ary word is a sequence of $n$ symbols from $A$:  $w=w_1w_2\dots w_n$ 
        where $w_i\in A$.
        \item A q-ary \emph{block code} of length $n$ over $A$ is a non-empty set $C$ of q-ary words
        having the same length $n$.
        \item An element in $C$ is called a \emph{codeword} in $C$.
        \item The number of codewords in $C$ is called the \emph{size} of $C$ and is denoted by $|C|$.
        \item A code of length $n$ and size $M$ is called an $(n,M)$-code.
    \end{enumerate}
\end{definition}

For instance, $C=\{000, 111, 010, 101\}$ is a block code of length $3$ over the binary alphabet $\{0,1\}$,
and q-ary of size $3$ each. This is a $(3,4)$-code.

\begin{definition}
A \emph{communication channel} consists of a finite channel alphabet $A=\{a_1, a_2, \dots a_q \}$, 
as well as a \emph{forward channel probabilties} $\Pr[a_j\text{ received}| a_i\text{ sent}]$, satisfying
$$ \sum_{j=0}^q \Pr[a_j\text{ received}| a_i\text{ sent}] = 1$$

for each different $i$.
\end{definition}

In other words, a communication channel allows specific symbols to be sent, but when sending something 
something else might arrive (with some probabiltiy).

\begin{definition}
    A communication channel is said to be \emph{memoryless}  if the outcome
    of any one transmition is independent of the outcome of the previous 
    transmitions.
    I.e., if $c=c_1c_2\dots c_n$ and $x=x_1x_2\dots x_n$ are words of length $n$ then 
    $$ \Pr[x \text{ received}| c \text{ sent}] = \prod_{i=1}^n \Pr[x_i \text{ received}| c_i \text{ sent}]$$
\end{definition}

\begin{definition}
a q-ary \emph{symmetric channel} is a memoryless channel such that 
\begin{enumerate}
    \item Each symbol trasmitted has the same probability $p$ of being received in error.
    \item if a symbol is received in error, then it is equally likely to be received as any of the other $q-1$ symbols.
\end{enumerate}
\end{definition}

For instance, let's look at the \emph{Binary symmetric channel} (BSC) with error probability $p$.
The channel alphabet is $\{0,1\}$, and the forward channel probabilities are:
\begin{itemize}
\item $ \Pr[0\text{ received}| 0\text{ sent}] = 1-p$
\item $ \Pr[1\text{ received}| 0\text{ sent}] = p$
\item $ \Pr[1\text{ received}| 1\text{ sent}] = 1-p$
\item $ \Pr[0\text{ received}| 1\text{ sent}] = p$
\end{itemize}

Given such a channel one can create a code that relies on the channel's properties to correct errors.
For instance, the Maximum likelihood decoding algorithm (which I skip).


\subsection{Distance of a code}

We use distances of codes to measure how well they can correct and detect errors.
The most common distance used is the Hamming distance.

\begin{definition}
    Let $x$ and $y$ be words of the same length $n$ over an alphabet $A$.
    The \emph{(Hamming) distance} between $x$ and $y$ is the number of positions in which they differ.
    If $x=x_1x_2\dots x_n$ and $y=y_1y_2\dots y_n$ then the distance is defined as:
    $$ d(x,y) = |\{i: x_i\neq y_i\}|$$

    where $x_i$ and $y_i$ are regarded as words of length $1$.
\end{definition}

For instance, the distance between $101$ and $111$ is $1$, 
and the distance between $101$ and $000$ is $3$.

Few properties of the Hamming distance:
\begin{itemize}
    \item $d(x,y)\ge0$ and $d(x,y)=0$ if and only if $x=y$ (non negativity).
    \item $d(x,y)=d(y,x)$ (symmetry).
    \item $d(x,y)\le d(x,z)+d(z,y)$ (triangle inequality).
\end{itemize}

Given a code block $C$, one can develop a decoding mechanism
based on the Hamming distance, known as the \emph{nearest neighbor decoding} (which I skip).


\begin{definition}
    The \emph{minimum distance} of a code $C$ is the smallest distance between any two distinct codewords in $C$.
    It is denoted by $d(C)$.
    I.e., $d(C) = \min\{d(x,y): x,y\in C, x\neq y\}$.
\end{definition}

\begin{definition}
    An $(n,M)$-code $C$ with minimum distance $d$ is called an $(n,M,d)$-code.
\end{definition}


\begin{example}
    $C=\{00000, 00111, 11111\}$ is a $(5,3,2)$-code. (since the minimal distance is $2$).
\end{example}


It turns out that the distance of a code is intimately related to the error-
detecting and error-correcting capabilities of the code.

\begin{definition}
Let $u$ be a positive integer. A code $C$ is $u-error-detecting$ 
if, whenever a codeword receives at least 1 error but no more than $u$ results
in a word that is not a codeword.
\end{definition}

\begin{example}
    The set $C=\{00000, 00111, 11111\}$ is a $1-error-detecting$ code because a single bit flip
    cannot result in a codeword being changed. 
    However, it is not a $2-error-detecting$ code because changing $00111$ to $11111$ is possible with 
    only two bit flips.
\end{example}

\begin{theorem}
    A code $C$ is $u-error-detecting$ if and only if the minimum distance of $C$ is at least $u+1$.
    (i.e., $d(C)\ge u+1$).
\end{theorem}
We skip the proof.


We have the same notion for error detecting codes, but for error correcting codes.
\begin{definition}
Let $v$ be a positive integer. A code $C$ is $v-error-correcting$
if minimum distance decoding is able to correct $v$ or fewer errors.
\end{definition}

\begin{example}
    Consider the binary code $C = \{000, 111\}$. By using the mini-
mum distance decoding rule, we see that
\begin{itemize}
    \item If $000$ is sent and one error occurs in the transmission, then the received word
    ($100$, $010$ or $001$) will be decoded to $000$.
    \item If $111$ is sent and one error occurs in the transmission, then the received word
    ($011$, $101$ or $110$) will be decoded to $111$.
\end{itemize}

In all cases 

\end{example}