
\chapter{Algorithms}
\section{FFT, DFT and NTT}
\subsection{representations of a polynomial}
we can either use \emph{coefficient representation} or \emph{Point-value representation}.

\subsubsection*{coefficient representation.}
A coefficient representation of a polynomial $P(x)$ is a vector of coefficients 
$a=(a_0,a_1,\dots, a_{n-1})$.
In coefficient representation, we can evaluate or add a polynomial in $O(n)$ time. 
unfortunately multiplying two polynomials takes $O(n^2)$.
The operation of multiplying two polynomials $a,b$ is called convolution $c=a\otimes b$.


\subsubsection*{Point-value representation.}
A point-value representation of a polynomial $P(x)$ of degree-bound $n$ is a set of $n$ point-value pairs.
$$ \{ (x_0,P(x_0)),\dots , (x_{n-1}, P(x_{n-1})) \}$$

\begin{theorem}[Uniqueness of an interpolating polynomial]
  For any set $ \{ (x_0,P(x_0)),\dots , (x_{n-1}, P(x_{n-1})) \}$ of $n$
  point-value pairs such that all the $x_k$ values are distinct,
  there is a unique polynomial $P(x)$ of degree-bound $n$ such that
  $y_k = P(x_k)$. for $k\in[n]$ 
\end{theorem}

Due to the above theorem, we can see that this representation defines a polynomial well. 
And one can move between both representations.

Because $C(X)=A(X)B(X)$ for any $x_j$, then $C(x_j)=A(x_j)B(x_j)$. Meaning 
multiplication and addition in this representation cost $O(n)$.
 
\subsection{FFT}
Luckily there is a way to move to and from both representations quickly:
\subsubsection*{root of unity}
A complex $n$th root of unity is a complex number $\omega$ such that
$\omega^n =1$.
There are exactly $n$ roots of unity: $e^{2\pi i k/n}$ for $k\in[n-1]$ 
($e^{iu}= \cos{u}+i \sin{u}$).

The value $\omega_n=e^{2\pi i/n}$ is the \emph{principal nth root of unity}.
all other complex nth roots of unity are powers of $\omega_n$:
 $\omega_n^0, \omega_n^1, \omega_n^2,\dots, \omega_n^{n-1}$.

FFT takes special advantage that these values have special properties.
\begin{lemma}[roots of unity]\label{roots-of-unity-property}
If $z$ is an nth root of unity, then $ a \equiv b \mod n \Rightarrow z^a = z^b$.
\end{lemma}
meaning that $\omega_n^a = \omega_n^b$. one can label specific points on the unit circle over $\C$.

\paragraph{Why is that important?}

In the simplest of cases, we are going to evaluate the polynomial over the following
points $ \{\omega_n^0, \omega_n^1, \omega_n^2,\dots, \omega_n^{n-1}\}$.
Let us view the polynomial evaluation over $n$ such points as a matrix, this presentation
might help the reader see some similarities:


$$
\begin{pmatrix}
  y_0\\
  y_1\\
  \vdots \\
  y_{n-1}\\
\end{pmatrix}
=
\begin{pmatrix}
  1 & x_0 & x_0^2 ,\dots\\
  1 & x_1 & x_1^2, \dots,\\
  \vdots & \vdots & \vdots & \ddots & \\
  1 & x_{n-1} & x_{n-1}^2, \dots,\\
 \end{pmatrix}
 \cdot 
\begin{pmatrix}
  a_0\\
  a_1\\
  \vdots \\
  a_{n-1}\\
\end{pmatrix}
 $$

 Now, let us assign the values:

 \label{dft-matrix}
 $$
\begin{pmatrix}
  p(\omega_n^0)\\
  p(\omega_n^1)\\
  \vdots \\
  \omega_n^2\\
\end{pmatrix}
=
\begin{pmatrix}
  1 & (\omega_n^0)^1 & (\omega_n^0)^2 ,\dots ,(\omega_n^0)^{n-1}\\
  1 & (\omega_n^1)^1 & (\omega_n^1)^2, \dots, (\omega_n^1)^{n-1}\\
  \vdots & \vdots & \vdots & \ddots & \\
  1 & (\omega_n^{n-1}) & (\omega_n^{n-1})^2, \dots,(\omega_n^{n-1})^{n-1}\\
 \end{pmatrix}
 \cdot 
\begin{pmatrix}
  a_0\\
  a_1\\
  \vdots \\
  a_{n-1}\\
\end{pmatrix}
 $$

 The keen looker can see that we can reuse values in this matrix.
 for example, let us look at $(\omega_n^{n-1})^{n-1}$ using \ref{roots-of-unity-property}
 $(\omega_n^{n-1})^{n-1} = (\omega_n)^{n*n-2n+1}=(\omega_n)^{0+0+1}$.  

 Thus we gain a matrix with heavy symmetry, which the FFT can use.



\subsubsection*{fft main idea}
The FFT method employs a divide-and-conquer strategy, 
using the even-indexed and odd-indexed coefficients of $A(x)$
separately to define the two new polynomials $A^{[0]}(x)$ and $A^{[1]}(x)$ of degree-bound $n/2$.

\begin{align}
  &A^{[0]}(x) =  a_0 +a_2x + a_4x^2 + \dots + a_{n-2}x^{n/2-1}\\
  &A^{[1]}(x) =  a_1 +a_3x + a_5x^2 + \dots + a_{n-1}x^{n/2-1}.
\end{align}


It follows that $A(x)=A^{[0]}(x^2) + x\cdot A^{[1]}(x^2)$.

Thus, the problem of evaluating $A(x)$ at 
$\omega_n^0, \omega_n^1, \omega_n^2,\dots, \omega_n^{n-1}$ 
reduces to evaluating the degree-bound $n/2$ polynomials  $A^{[0]}(x)$ and $A^{[1]}(x)$ at the points
$$(\omega_n^0)^2, (\omega_n^1)^2,\dots, (\omega_n^{n-1})^2$$
and then combining the results according to $A(x)=A^{[0]}(x^2) + x\cdot A^{[1]}(x^2)$.


Using the properties of the root of unity, 
the above values consist not of $n$ distinct values but only of the $n/2$
 complex ($n/2$)th roots of unity, with each root occurring exactly twice!
These subproblems have the same form as the original problem but are half the size.
We can continue in this manner until we evaluate the full polynomial on $n$ distinct values.


\subsection{putting it all together}
As we've seen we can create a matrix with heavy symmetry, called the DFT matrix. 
now, we'd want to capitalize on that and create a recursive algorithm.
To do so, instead of the regular polynomial assignment we've seen above using a matrix,  
we'll shuffle things a bit according to the divide-and-conquer approach from above
$$
F_n \cdot p(x) = 
\begin{pmatrix}
  I_{ \frac{n}{2}} & D_{\frac{n}{2}} \\
  I_{ \frac{n}{2}} & -D_{\frac{n}{2}}
 \end{pmatrix}
 \cdot 
 \begin{pmatrix}
  F_{ \frac{n}{2}} & 0 \\
  0 & F_{ \frac{n}{2}} 
\end{pmatrix}
\cdot 
\begin{pmatrix}
  p_{\text{even coefficient}} \\
  p_{\text{odd coefficient}}
\end{pmatrix}
 $$

 where $F_n$ is the matrix of assignment we've seen in ~\ref{dft-matrix}, and 
 $D_n$ is the following diagonal matrix $$
 \begin{pmatrix}
  1 & 0 &0  & \dots& 0 \\
  0 & \omega_n^1 &0  & \dots& 0 \\
  1 & 0  &\omega_n^2  & \dots& 0 \\
  \vdots & \vdots & \vdots & \vdots &  \\
  0 & 0  &0  & \dots& \omega_n^n-1 \\
 \end{pmatrix} $$


 We can continue this process recursively, to get what we want.
 In each recursive level, the algorithm does at most $O(n)$ work (i.e., multiplying by sparse matrices),
  and due to the logarithmic depth of the recursion, we get $O(nlogn)$.
  \begin{lstlisting}[language=Python,
    caption={fft algorithm in python}
    ]
import numpy as np

def fft(p: np.ndarray) -> np.ndarray:
    n = len(p)
    if n == 1:
        return p
    omega = np.exp(2 * np.pi * 1j / n)

    p_even, p_odd = p[::2], p[1::2] 
    ye, yo = fft(p_even), fft(p_odd)

    # mult the structured left matrix with the recursive result.
    # O(n) work comes from here.
    y = np.zeros(n, dtype=np.complex128)
    for k in range(n // 2):
        y[k] = ye[k] + omega ** k * yo[k]
        y[k + n // 2] = ye[k] - omega ** k * yo[k]
    return y
    \end{lstlisting}
    
  
\subsection*{Interpolation}

The process is similar, we simply need need to invert $F_n$~\ref{dft-matrix}.
% We can write the polynomial as a Vandermonde matrix:
% $$
% \begin{pmatrix}
%   y_0\\
%   y_1\\
%   \vdots \\
%   y_{n-1}\\
% \end{pmatrix}
% =
% \begin{pmatrix}
%   1 & x_0 & x_0^2 ,\dots\\
%   1 & x_1 & x_1^2, \dots,\\
%   \vdots & \vdots & \vdots & \ddots & \\
%   1 & x_{n-1} & x_{n-1}^2, \dots,\\
%  \end{pmatrix}
%  \cdot 
% \begin{pmatrix}
%   a_0\\
%   a_1\\
%   \vdots \\
%   a_{n-1}\\
% \end{pmatrix}
%  $$

%  and then if we assign the roots of unity as $x_0,x_1,\dots$
%  we should be able to compute the inverse of the above matrix very quickly
%  (they have some proof for that).
 
%  Then there is a quick method for computing $y\cdot Matrix(A(X))$.

 \begin{corollary}
  For FFT, DFT, and NTT we must use specific values to gain speedy interpolation and
  evaluation of the polynomial. 
 \end{corollary}{Conclusin:}
 
\subsection{NTT}
The NTT is a specialized version of the discrete Fourier transform,
in which the coefficient ring is taken to be a finite field (or ring)
 containing the right roots of unity.
It can be viewed as an exact version of the complex DFT, avoiding 
round-off errors for exact convolutions of integer sequences.
While Gauss used similar techniques already in~\cite{gaussfft}, laying the groundwork 
for modern FFT algorithms to compute the DFT, and therefore the NTT, 
is usually attributed to Cooley and Tukey's seminal paper \cite{fft}.

\subsubsection{Special Features}
Applying the NTT transform provides a cyclic convolution, computing 
$c = a \cdot b \mod (X^n + 1)$  with two polynomials $a$ and $b$ would require 
applying the NTT of length $2n$ and thus $n$ zeros to be appended to each input;
this effectively doubles the length of the inputs and also requires the computation of 
an explicit reduction modulo $X^n +1$. 
To avoid these issues, one can exploit the \emph{negative wrapped convolution} \cite{negantt}.

This exploitation allows us to perform $NTT$ without expanding to $2n$ parameters!
meaning we can send less bandwidth over the network, and perform fewer computations too. 
This trick can be seen in \cite{SEAL}.

\section{Reed-Solomon}
It is fairly simple, but effective algorithm to encode words and be able to deal with erasurs.
\subsection{Naive solution}
assuming we decide we want our words to be of size $k=4$, and we add additional two redundancies $n=6$. 
Thus to send it we'll do the following: given a word vector, we'll treat it as coefficients values of a 
polynomial of degree $k-1$ (because we have k coefficients its a k-1 degree poly), then to encode it, we'll evaluate it on the following $x$ values: $(-1,0,1,2,\dots n)$.


Now given the 6 values that we received over the network we can decode it as follows: 
We go over every group of 4 points, and try to reconstruct the polynomial. 

the polynomial we've reconstructed the most is the correct one.

\subsection{Systematic Reed-Solomon}
Assuming we want to be able to correct up to $s$ values, we'd need at least $2s$ redundant symbols.
So if $k=4$, and $n=6$, we can only recover a single byte.

It won't work for larger values, like $k=223$, $n=255$ which is a popular value for reed-solomon.
(Meaning we add 32 more symbols to our message, so we can repair 16 symbols in total).

The previous algorithm will work, but we'd need to send different values then 
the original message. It would be best to send $Message||\text{addition code word}$.
that way,we might be able to avoid reconstruction if we can find quickly that there was no error.

this is called a Systematic code. 
