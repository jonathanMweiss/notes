
\begin{defn}[$\lambda$-bit strength]
  Defining the strength of security is complex. NIST decided that $\lambda$-bit security 
  means that for any kind of known attack, the attacker would need at least $2^{\lambda}$
  operations to break the scheme.

  Where each operation is defined as a single op over the specified field or ring.
  For instance, over an elliptic curve, either multiplication or additions are assumed to 
  take 1 single CPU cycle.
  
\end{defn}

\chapter{Basics}
\section{Diffie-Helman}
Diffie-Helman relies on the following assumption.

\begin{defn} [Decisional Diffie-Helman (DDH)]\label{DDH}
  Consider a (multiplicative) cyclic group $\G$ of order $q$ and with generator $g$. 
  Given $g^a$ and $g^b$ for $a,b\in_R Z_q$ then $(g^a,g^b,g^{ab})$, $(g^a,g^b,g^{ab})$ are computationally indistinguishable:

  $$ \Pr[A((g^a,g^b,g^{ab}))=1] - \Pr[A(g^a,g^b,g^{c})=1]\le \delta(n)$$
  for $\delta(n)$ a negligable function in $n$.
\end{defn}

Given this assumption, which most of the world takes as FACT (until proven otherwise), then
we can define the key exchange protocol.
\paragraph{DH key exchange}
While not safe against man-in-the-middle attacks, DH is an excellent way to create a shared secret.
The protocol goes as follows.
Assume we have two players, Alice and Bob, both know the other's public key $g^a$ and $g^b$ respectfully,
each takes their secret key $a$ or $b$ and raises the other's public key to the power of $a$ (or $b$).
Both now hold $g^{ab}$, and no one else can compute it since they don't know the secret of either Bob or Alice.

Alice and Bob can use this new number/ or elliptic curve point $g^{ab}$ as a private key for
a symmetric encryption scheme like AES (e.g., $symKey=Hash(g^{ab})$).


\chapter{Elliptic curves}
Taken from \emph{A Graduate Course in Applied Cryptography, Dan Boneh and Victor Shoup}

The best-known discrete log algorithm in an elliptic curve group of size $q$ runs in time $O(\sqrt{q})$. 
This means that to provide security comparable to AES-128, it suffices to use a group of size $q \approx 2^{256}$ 
so that the time to compute discrete log is $\sqrt{q} \approx 2^{128}$.

\section{a bit of history on elliptic curves}
Elliptic curves come up naturally in several branches of mathematics.
Here we will follow their development as a branch of arithmetic (the study of rational numbers).
Diophantus, a greek mathematician who lived in Alexandria in the third century, was interested 
in the following problem: Given a bivariate polynomial equation, $f(x,y)=0$ find 
rational points satisfying the equation. That is, find $P=(x,y)$ such that $x,y\in\mathbb{Q}$.
Diophantus wrote a series of influential books on this subject, called the \emph{Arithmetica}, of which six survived.
Much of Arithmathica studies integer and rational solutions of quadratic equations.
For example, one of the first problems in the books regarding elliptic curves is to find $(x,y) \in \Q$
 that satisfy the equation $$ y^{2}=x^3 - x + 9$$
The method invented to answer this question secures most Internet traffic worldwide.
One can easily verify that the following points are on the curve $(0, \pm3), (1, \pm3), (-1, \pm 3)$.
Diophantus developed a method similar to \underline{chord method}:
Given two rational points $U$ and $V$ on the curve, where $U\neq -V$, we can pass a line through them,
 and this line must intersect the curve at a third rational point $W$. 
The chord method was rediscovered several times over the centuries, but it finally stuck with the work of Poincare.

Poincare defines the sum of $U, V$, denoted $U \boxplus V$ as $U \boxplus V := -W$
Diophantus' method, the \emph{tangent method} is another way to build a new rational point from a given rational point. 
We should later see that the method corresponds to adding the point $P$ to itself: $P\boxplus P$.
\section{elliptic curves over finite fields}
We are primarily interested in elliptic curves over finite fields for cryptographic applications.
\begin{defn}
  Let $p >3 $ be a prime. An \textbf{elliptic curve} $E$ defined over $\mathbb{F}_p$ is an 
  equation  
  $$ y^2 = x^3 +ax + b$$
  where $a,b \in \mathbb{F}_p$ satisfy $4a^3 +27b^2 \neq 0$. We write $E/\mathbb{F}_p$ to denote
  the fact that $E$ is defined over $\F_p$.
\end{defn}
The weird condition $4a^3 +27b^2 \neq 0 $ ensures that the equation 
$ x^3 + ax + b = 0$ does not have a double root (needed to avoid degeneracies).

\begin{defn}
  We say that point $P=(x,y)$, where $x,y \in \F_p$, is on curve $E$ if it satisfies the equation of $E$.
\end{defn}
\begin{defn}
  $E(\F)$ denotes the set of all points on the curve E and are defined over $\F$.
\end{defn}

The addition law, and adding to the set $E(F_p)$ the member $\mathcal{O}$,
 which servers as the identity member, turns it into a group. I'm not getting into the addition laws.

\section{Elliptic curve cryptography}


Given two points $P$ and $\alpha \cdot P$, where $\alpha \in \Z_q$, it is hard to compute $\alpha$.
The best known algorithm to compute $\alpha$ is $\Omega(\sqrt{q})$.
There are a few exceptions to these rules, and thus one should be careful when defining a new curve.
So it's better to use some specific already implemented and used curve rather than 
generating a random prime $p$ and a random curve over $\F_p$.
\todo{TODO, put here an encryption scheme, and a signature scheme (can use schnorr here honestly)
The most important thing to remember is that elliptic curve cryptography 
assumes hard exponent, and everything is similar to those algorithms
}

\subsection{ECDSA}
The ECDSA is essential DSA over elliptic curves.
The most commonly known is the signature algorithm used by bitcoin and other cryptocurrencies:
ECDSA, which commonly refers to the following algorithm:

\begin{description}
  \item[Key generation] Grab a random number $x\in \mathbb{F}_q$ where the elliptic curve  
  is a vector space over $\mathbb{F}_q$, and $q$ is the order of the group.
  and return $(sk,pk)=(x,xG)$, where G is the generator of the elliptic curve 
  (Usualy $G$ is a specific base point, because over a a group with $q$ points, 
  almost every point is a generator - due to Fermat's little theorm \ref{fermats-little-theorem}).
  \item[Signing] $(r,s)=\sigma = Sig_{sk}(m)$ is generated as follows:
  \begin{enumerate}
    \item generate a random scalar $k\in \mathbb{F}_q$.
    \item compute $R=g^{k^{-1}}$
    \item compute $r=H(R)$ (notice that this is now a scalar).
    If $r mod=0$ then go back to step 1.
    \item compute $s=k^{-1}\cdot(H(m)+x\cdot r)$\todo{This part is incorrect because
     they want specific bits from $H(m)$, but it's not crucial to understand the algorithm,
      so I skipped it.}
    \item set $\sigma=(r,s)$.
  \end{enumerate}
  \item[Verification] To verify a signature $\sigma=(r,s)$ on message $m$ with 
  public key $pk$, one needs to compute the following 
  $ R'= G^{m\cdot s^{-1}}\cdot pk^{r\cdot s^-1} $ (note that this is a point in the EC).
  Notice that if you subtitute $s$ and $pk$ to their actual values, you should get $R$ if 
  the signature is valid.
  Then, check that $r=H(R')$, if it is accept the signature.
\end{description}

\subsection{Schnorr Signature}
\todo{need to fill this...}
\section{Pairing based cryptography}
We now show that certain elliptic curves have an additional structure,
called a pairing, which enables a world of new schemes that could not be built
 from discrete log groups without this other structure.

 \begin{defn}
  Let $\G_0, \G_1, \G_T$ be three cyclic groups of prime order $q$ where 
  $g_0\in\G_0$ and $g_1\in\G_1$ are generators. A \textbf{pairing} is an efficiently
  computable function $e: \G_0 \times \G_1 \to \G_T$ satisfying the 
  following properties:
  \begin{itemize}
    \item bilinear: $\forall u,u'\in\G_0, v,v'\in\G_1$  we have 
        $$ e(u\cdot u',v) = e(u,v) \cdot e(u',v) and  e(u,v\cdot v') = e(u,v)\cdot e(u,v')$$
    \item non-degenerate: $g_T := e(g_0,g_1)$  is a generator of $\G_T$.
  \end{itemize}
\end{defn}
We refer to $\G_0,\G_1$ as the pairing groups and $\G_T$ as the target group.

Bilinearity implies the following central property of pairing that will be used
in all our constructions: for all $a, b\in \Z_q$ we have
  $$ e(g_0^a,g_1^b)=e(g_0,g_1)^{a\cdot b}= e(g_0^b,g_1^a)$$
This equality follows from the equalities 
$e(g_0^a,g_1^b)=e(g_0,g_1^b)^a=e(g_0^a,g_1^b)=e(g_0^a,g_1)^b$
which are themselves a direct consequence of bilinearity (Note that in cyclic groups
 such as $\G_0 and \G_1$ this definition is equivalent to the definition of bilinearity above).
[if I understand correctly, due to $\G_0 and \G_1$ having generators, we can represent every number
as a power of the generators themselves, and using bilinearity we can reach the above equations]

In a symmetric pairing group, where $G_0 = G_1$ schemes like ElGamal aren't secure anymore,
that is, given two public keys, it is to break the DDH assumption: 
  $$ e(g^x,g^y) = e(g,g)^{x\cdot y}$$ 
Hence, it is possible to extract information. 

In asymmetric pairing, where $G_0 \neq G_1$, it may still be possible
that the DDH assumption holds in both $G_0$ and $G_1$. 

Furthermore, if the discrete log in $G_T$ is easy, it is easy in $G_0$ (and $G_1$).

\begin{rem}
  The pairing function $e$ on an elliptic curve comes from an algebraic pairing called the \textbf{Weil pairing}.
  Which can be efficiently evaluated. In practice, one uses variants of the weil 
  pairing, called the Tate and Ate pairings.
\end{rem}

We'll avoid $bn256$ curve as it isn't secure.


\subsection*{optimisations}
\begin{itemize}
  \item When the first value of the pairing is known, we can speed things up
  by calculating a table of the other possible values. (though it seems very inefficient, and memory intensive.)
\end{itemize}

\section{Advanced encryption schemes from pairings}

\subsection{The BLS signature scheme}
Let $e:F_0 \times G_1 \to G_T$ be a pairing where $G_0,G_1,G_T$ are
cyclic groups of prime order $q$, and where $g_0\in G_0, g_1\in G_1$ are
generators. 
We'll also require a hash function $H : \mathcal{M} \to G_0$.

The BLS signature scheme denoted $\mathcal{S}_BLS = (G,S,V)$, has message space
$\mathcal{M}$ and works as follows:
\begin{itemize}
  \item $G():$ The key generation algorithm runs as follows 
  $$  \alpha \from \Z_q,u\from g_1^\alpha \in G_1  $$
  The public key is $ pk := u$, and the secret key is $\alpha$.

  \item $S(sk,m):$ To sign a message $m\in \mathcal{M}$ using a secret key
  $sk=\alpha\in\Z_q$ do:
  $$ \sigma\from H(m)^{\alpha} \text{ output } \sigma$$

  \item $V(pk,m,\sigma):$ To verify a signature 
  $$ e(H(m), u) = e(\sigma,g_1) $$
  that is, the signature is accepted due to the following calculations
  $$e(H(m), g_1^\alpha) = e(H(m),g_1)^\alpha = e(H(m)^\alpha,g_1)= 
  e(\sigma, g_1)
  $$
\end{itemize}

\subsection{Identity based encryption}
Using Identity-Based Encryption (IBE from now), Alice can send an encrypted message to Bob
without asking for bobs key beforehand. Beyond key exchange, IBE can also 
be used to construct CPA-secure public key encryption and even search encrypted data.
In an IBE scheme, a trusted entity, Trudy, generates a master key pair $(mpk, msk)$.
The key $mpk$ is the \emph{master public key} and is known to everyone. Trudy keeps the \emph{master 
secret key} $msk$ to herself. When Bob wants to use his email address as his public key, he must
somehow obtain the private key derived from $msk$ and Bob's public key.
We denote Bob's email address as his (public) identity $id$ and denote the corresponding private key 
by $sk_{id}$. He receives $sk_{id}$ by contacting Trudy, she'll use $msk$ and $id$ to generate $sk_{id}$,
and send it back to Bob.
Now anyone, including Alice, can send bob messages without looking up his public key. 
Assuming, of course, that Alice knows $mpk$.

\begin{defn}
  An \textbf{Identiy based encryption scheme $ \mathcal{E}_id =(S,G,E,D)$}
  is a tuple of four efficient algorithms: a \textbf{setup algorithm $S$}, a 
  \textbf{Key generation algorithm $G$}, an \textbf{encryption algorithm $E$}
  and a \textbf{decryption algorithm $D$}.

  \begin{itemize}
    \item $S$ is a probabilistic algorithm which outputs $(mpk, msk) \overset{_R}{\leftarrow} S()$.
    \item $G$ is a probabilistic algorithm invoked as $sk_{id} \getrandom G(msk, id)$.
    \item $E$ is a \emph{probabilistic} algorithm invoked as $c \getrandom E(mpk, id, m)$.
    \item $D$ is a deterministic algorithm invoked as $m\getrandom D(sk_{id},c)$. Where $m$ is
      either a message or a \textbf{reject} value.
    \item We require that decryption undoes encryption: 
    $$ \Pr[D(G(msk,id), E(mpk,id,m))=m]=1$$
    \item identities, messages, and ciphertexts are in their respective finite spaces,
    and $\mathcal{E}_id$ is defined over $(\mathcal{ID},\mathcal{M},\mathcal{C})$.
  \end{itemize}
\end{defn}

We can view IBE as a particular public key encryption scheme where the messages
to be encrypted pairs $(id, m)\in\mathcal{ID\times M}$. The master secret key can
decrypt any well-formed ciphertext. However, there are weaker secret keys, 
such as $sk_{id}$ that can decrypt a ciphertext $c\getrandom E(mpk, id', m)$ only if $id'=id$.


Bob won't use his email address as a public key if a company uses IBE. Otherwise, if Bob's secret key 
is compromised, bob would need to change his email. Instead, Bob's public key
should bob could use the identity string $(id, date)$.
Bob will need to request a new secret key from Trudy every day.
Now Alice can even encrypt emails for specific dates, and bob cannot access it before that, 
unless Trudy gives bob any key he requests.

I will not go into detail about the formal definition of security, but the idea is that
the attacker has access to any $sk_{id'}$ it want but should not be able to break semantic
security for some other identity with secret key $sk_{id}$ that the adversary does not have.


\subsection*{IBE from pairings}
We'll need a few components before being able to construct an IBE: 
\begin{itemize}
  \item a pairing. as defined above, over cyclic groups of prime order $q$, 
  with generators $ g_0, g_1 $.
  \item a symmetric cipher $\mathcal{E}=(E_s,D_s)$.
  \item hash functions $H_0:\mathcal{ID}\to \G_0$ and $H_1:\G_1\times\G_T\to \mathcal{k}$,
  where k is the key space.
\end{itemize}
\subsubsection*{Construction 1}

\begin{itemize}
  \item $S():$ runs as follows:
  $$ \alpha\getrandom \Z_q, u_1\from g^\alpha, mpk \from u_1, msk \from \alpha 
  \text{ output }(mpk,msk) $$
  \item $G(msk,id):$ key generation using $msk = \alpha$:
  $$ sk_id \from H_0(id)^\alpha \in \G_0 \text{ output } sk_id$$
  \item $E(mpk, id, m ):$ encryption using the public parameters $mpk=u_1$:
  \begin{align*}\label{lamlam} 
    \beta \getrandom \Z_q,\  z\from e(H_0(id),mpk^\beta) 
    \\ \ 
    k\from H_1(g^\beta,z), c\getrandom E_s(k,m), \text{ output } (g^\beta,c)
  \end{align*}
  The encryption generates a new secret key $k$ at random. This private key relies on the pairing with
  the hashed public id. Once we've generated the ephemeral secret key $k$, we encrypt the message $m$.

  \item $D(sk_id,(w_1,c)):$ decryption using secret key $sk_id$ of ciphertext $(w_1,c)$:
  \begin{align*}
    z \from  e(sk_{id},w_1), k\from H_1(w_1, z), m\from D_s(k,c),
  \text{ output } m
  \end{align*}
  The decryption relies on $w_1=g^\beta$ to figure out the symmetric encryption key $k$.
  With $k$, the message can be decrypted.
  \newline
  To generate $k$ correctly, we first remind the reader that $$sk=H(id)^\alpha, mpk^\beta=g^{\alpha\beta}, k=H(g^\beta,z)$$
  now, let's compute $z$, our missing component:
  $$
  z=\underbrace{e(sk_{id},g^\beta)}_{\text{z in decryption}}=e(H(id)^\alpha,g^\beta)=e(H(id),g^{\alpha\beta})=\underbrace{e(H(id),mpk^\beta)}_{\text{z in encryption}}
  $$
\end{itemize}


\chapter{Lattices}
The following chapters are chosen parts from Peikert's Survey, and the notes from the MIT course Advanced Topics in Cryptography: From Lattices to Program Obfuscation.

\section{Basic definition}

Latices are n-dimensional subsets of $\mathbb{R}^n$. 
A lattice $\mathcal{L}$ is both 
\begin{itemize}
    \item an \emph{addative} subgroup: $0\in \mathcal{L}$, and for every $x,y\in \mathcal{L}$ so are $-x, x+y\in \mathcal{L}$.
    \item \emph{discrete}: every $x\in \mathbb{L}$ has a ball/ neighbourhood in $\mathbb{R}^n$, where $x$ is the only lattice point. 
\end{itemize}

for instance $\mathbb{Z}^n$ is a lattice, or just the even numbers is a lattice. In addition, given a lattice $\mathcal{L}$, for any $c\in \mathbb{R}$, $c\cdot\mathcal{L}$ is also a lattice.

Every lattice has a base (since it is a vector space).

Every lattice has a dual: $\mathcal{L}^*=\{w : <w,\mathcal{L}>\subseteq \mathbb{Z}\}$


\begin{definition}
    The minimum distance of a lattice $\mathcal{L}$ is the length of a shortest nonzero lattice vector:
    $$ 
    \lambda_1(\mathcal{L}) := \min_{v\in \mathcal{L} \setminus \{0\}}{||v||} 
    $$
\end{definition}
\subsection{Problems Over Lattices}
Apperently it is not easy to solve things over lattices. 

\begin{definition}[Shortest Vector Problem (SVP)]
    Given an arbitrary basis $B$ of some lattice $mathcal{L}= mathcal{L}(B)$ ($mathcal{L}$ is defined via the basis $B$),
find a shortest nonzero lattice vector, i.e., a $v \in mathcal{L}$ for which $||v|| = \lambda_1(\mathcal{L})$.
\end{definition}
Think about this one, like given a random basis of lattice vectors. It is hard to find all the vectors around the origin. 
Treat it like other problems that once they are discrete become hard (linear programming for instance).


There are many other problems, but i won't write about them.

\subsection{(Discrete) Gaussians and Subgaussians}
Discrete gaussian samples for  lattice are a cornerstone in many of 
lattice-based cryptographic primitives. But i won't write about it just yet.
treat it like a black box.



\section{Modern Foundations}
In this chapter we survey the main foundational works that directly underlie most modern lattice-based
cryptographic schemes. These include the two main average-case problems SIS and LWE, their analogues
over rings, and analytical techniques involving (discrete) Gaussian probability distributions
\subsection{Short Integer Solution (SIS)}
The short integer solution (SIS) problem was first introduced in the seminal work of Ajtai,and has
served as the foundation for one-way and collision-resistant hash functions, identification schemes, digital
signatures, and other "minicrypt" primitives (but not public-key encryption).

Here we define the SIS problem,
survey its connection to worst-case lattice problems, and explore some of its basic properties and immediate
cryptographic implications. (Further applications are described in~\cite[Chapters 5 and 6]{latticebook})


\begin{definition} [Short Integer Solution (${SIS}_{n,q,\beta,m}$)]\label{def:sis}
    Given $m$ uniformly random vectors $a_i\in \mathbb{Z}_q^n$ form-
ing the columns of a matrix $A\in \mathbb{Z}_q^{n\times m}$, find \emph{nonzero}
integer vector $z\in \mathbb{Z}^m$ of norm $||z|| \le \beta$ such that

$f_A(z):=Az=\sum_{i}{a_i \cdot z_i}=0\in \mathbb{Z}_q^n$
\end{definition}


We now highlight several simple but useful observations about the SIS problem:
\begin{itemize}
    \item without the norm constraints over $z$it is easy to find a solution via Gaussian elimination. Similarly, we
must take $\beta < q$, otherwise $z=(q,0,0,0,0,0...)\in \mathbb{Z}^m$ would be a legitimate (but trivial) solution.
\item any solution for $A$ can easily be extended to $A|A'$, by attaching zeros to $z$ (the norm remains the same). In other
words, we can ignore columns $a_i$ as desired, so the SIS problem can only become easier as $m$ increases. Similarly, it can only become harder as $n$ increases.
\item The norm should be large enough to guarantee a solution, which happens when $\beta  \ge \sqrt{\bar{m}}$ for $m\ge\bar{m}$ the smallest integer greater than $n\log q$. Why? Chris explains in chapter 4.
\item Additional point stating it is collision resistant.
\end{itemize}


\emph{SIS} is NP-Total. it has a solution, and easy to verify but hard to find a solution.
% TODO: fix following
\subsection{Learning With Errors}
A very important work of Regev \todo{cite! On lattices, learning with errors, random linear codes} from 2005 introduced the average-case learning with errors (LWE)
problem, which is the "encryption-enabling" analogue of the SIS problem.

The two problems are syntactically very similar, and can meaningfully be seen as duals of each other

\paragraph{defs:}
LWE is parameterized by positive integers $n$ and $q$, and an error distribution $\mathcal{X}$ over $\mathbb{Z}$. For concreteness, $n$
and $q$ can be thought of as roughly the same as in SIS, and $\mathcal{X}$ is usually taken to be a discrete Gaussian of
width $\alpha q$ for some $\alpha<1$, which is often called the relative "error rate".

\begin{definition}[LWE distribution] 
    for a vector $s\in\mathbb{Z}_q^n$ called the secret, the LWE distribution
    $A_{s,\mathcal{X}}$ over $\mathbb{Z}_q^n \times \mathbb{Z}_q$ sampled by choosing $a\in \mathbb{Z}_q^n$ uniformly at random, choosing $e\leftarrow\mathcal{X}$ and outputting $(a, <a,s>+e \mod q)$.
\end{definition}

There are two main versions of the LWE problem: \emph{search}, which is to find the secret given LWE samples,
and decision, which is to distinguish between LWE samples and uniformly random ones.

\label{def:LWE}
\begin{definition}[Search-LWE] 
    Given $m$ independent samples $(a_i,b_i) \in\mathbb{Z}_q^n \times \mathbb{Z}_q$ drawn from $A_{s,\mathcal{X}}$
for a uniformly random $s \in \mathbb{Z}^n_q$ (fixed for all samples), find $s$.
\end{definition}

Note that $b_i$ is derived using the secret and $A$, as defined by \emph{LWE distribution}.

\begin{definition}[decision-LWE] 
    Given $m$ independent samples $(a_i,b_i) \in\mathbb{Z}_q^n \times \mathbb{Z}_q$ where every sample is distributed according to either:  (1)
    $A_{s,\mathcal{X}}$ for a uniformly random $s \in \mathbb{Z}^n_q$ (fixed for all samples), or (2) the uniform distribution,
    
    Distinguish which is the case (with non-negligible advantage)
\end{definition}

Some highlights (like we did for SIS):
\begin{itemize}
    \item Without the error terms from $\mathcal{X}$ both problems are easy to solve, because we can efficiently recover s
from LWE samples by Gaussian elimination.
\item We can write instead of the $m$ samples $(a_i,b_i)$ simply: $b^t = s^t A +e^t$. where $e \leftarrow\mathcal{X}^m$.
\end{itemize}


\paragraph{\bf Normal Form} 
Similarly to SIS, the LWE problem also has a normal form, in which the coordinates
of the secret s are chosen independently from the error distribution $\mathcal{X}$.Using this form can
yield substantial efficiency gains for certain cryptographic constructions.

Pieker's survey discuss the hardness, but i skip it.


\subsection{Ring-SIS}
Micciancio introduced a compact ring-based analogue of Ajtai's SIS problem and its associated function $f_A$
from \autoref{def:sis}.

\paragraph{Definitions}
The ring-SIS problem is parameterized by:
\begin{itemize}
    
    \item A ring R, which is often (but not always) taken to be a degree-$n$ polynomial ring of the form
$R_q= Z_q[X]/(f(X))$, e.g., $f(X) = X^{2^k} + 1$. (For more information, see \autoref{def:polynomial-ring})
$R_q$ has the following norm: For a vector $z$ over $R_q$ we define $||z||=(\$sum_i{||z_i||^2})^{1/2}$.
    \item real norm bound $\beta > 0$ for the "short" solution, and a number $m$ of samples. (m is not super important)

\end{itemize}
    
\begin{definition}[Ring-Short-Integer-Solution (R-SIS)]\label{def:ring-sis}
    Given $m$ uniformly random elements $a_i \in R_q$, defining a vector  $a\in R^m_q$,
find a nonzero vector $z\in R^m$ (i.e., $R$ is like $R_q$ but without $\mod q$) of norm 
$||z||<\beta$ such that:

    $$f_a:=<a,z> = 0\in R_q$$
    

    note that $<a,z>$ is a vector multiplication between two vectors with polynomial entries.
\end{definition}

R-SIS is relatively more compact, where $m$ can be approximatly $log q$, instead of $n\log q$  as in the non-ring case.



\paragraph{Relation to SIS}
One can translate between cryptographic constructions over the two. 
read more in (4.3.2 Peikert survey).


\paragraph{\b One-wayness} For the ring $R= Z[X]/(X^n-1)$ and other appropriate parameters, Micciancio proved that
the function $f_a$ defined in \autoref{def:ring-sis} is one-way.



\subsection{Ring-LWE}

I will define the RLWE "decision" problem, since by using decisional-RLWE,
it is relatively easier to define encryption schemes under this notion of hardness.
Do note that there is an additional hardness assumption, called "search", 
which is equivalent under specific conditions and parameters to 
the "decisional-RLWE" problem.

For a full, and accurate definition see~\cite[def 3.3]{RLWE}.
\begin{definition} (toy decisional-RLWE)
    Let 
    \begin{enumerate}
        \item $a_i(x)$ be a set of random but known polynomials from $Z_q/\Phi(X)$ with 
        coefficients from $Z_q$.
        \item  $e_i(x)$ be a set of small random and unknown polynomials 
        relative to a bound $B$ in the ring $Z_q/\Phi(X)$.
        \item $s(X)$ be a small unknown polynomial 
        relative to a bound $B$ in the ring $Z_q/\Phi(X)$.
        \item $b_i(x)=a_i(x)\cdot s(x) + e_i(x)$
    \end{enumerate}
    
    Given a list of polynomial pairs $(a_i(x),b_i(x)$) it is computationally infeasible 
    to determine $s(x)$.
\end{definition}



\section{Essential Cryptographic Constructions}


\subsection{Collision-Resistant Hashing}
A collision resistant hashing scheme $\mathcal{H}$ consists of an ensemble of hash functions ${\mathcal{H}}n\in\mathbb{N}$ where each $\mathcal{H}_n$
consists of a collection of functions that map $n$ bits to $m<n$bits. So, each hash function $h$ compresses its
input, and by pigeonhole principle, it has collisions. That is, inputs $x\neq y$ such that $h(x)=h(y)$. Collision-
resistance requires that every p.p.t. adversary who gets a hash function $h \leftarrow \mathcal{H}_n$ chosen at random fails to find a collision except with negligible probability.

\paragraph{\bf Collision-Resistant Hashing from SIS}
Here is a hash family ${\mathcal{H}}$ that is secure under $SIS_{(n,q,\beta,m)}$
where $n \log q >  m \log (\beta +1)$. Each hash function $h_A$ is parameterized by a matrix 
$A\in \mathbb{Z}^{n\times m}_q$, takes as input $e\in [0,\dots,\beta]^m$ and outputs:

$$ h_A(e) = A\cdot e \mod q$$

\paragraph{\bf why is it secure?}
Assume we have two different values $x\neq y$ such that $x,y\in [0,\dots,\beta]^m$ and $h_A(x)=Ax=Ay=h_A(y)$, 
Why this means that $Ax-Ay=A(x-y)=0$ right? We managed to find a vector $z=x-y$ such that $f_A(z)=0 \mod q$, and since $z\in[-\beta,\dots,\beta]$ this is a solution to the SIS problem! (which is in contradiction to the hardness) .


\section{Fully Homomorphic Encryption}
Homomorphic encryption (\emph{HE}) is a form of encryption that allows 
computations to be performed on encrypted data without first having to decrypt it.
More formally:
\begin{definition}[HE]
 Let there be an encryption scheme $\pi = (KeyGen,Enc, Dec)$
 We say that $\pi$ is an homomorphic encryption scheme if
 given $f:\text{PTX-SPACE}\to \text{PTX-SPACE}$ then there
 exists $F:\text{CTX-SPACE} \to \text{CTX-SPACE}$ such that
 given two encrypted messages $c_1=Enc(m_1)$ $c_2=Enc(m_2)$ the following
 applies
   $Dec(F(c_1,c_2))=f(m_1,m_2)$
\end{definition}

That is, we can compute some wanted function $f$.

We have multiple encryption schemes. Some are \emph{partially HE},
like ElGamal, where only multiplication is possible.
\begin{theorem}
 We call the encryption scheme $\pi$ \emph{fully homomorphic encryption} (FHE) if it 
 can support addition, negation and multiplication.
\end{theorem}

As of the year 2025, most of the FHEs are heavy and slow,
or called leveled-FHE, where one must first define their parameters, 
and then it has a limited amount of operations to perform (many additions and a 
handful of multiplications.)

In the following sections, I'll talk about a specific scheme called BGV \cite{BGV},
this is a (leveled)-FHE scheme, which is easy to work with, and relatively efficient.

\subsection{Usages}
\red{TODO: describe PIR, private machine learning (for finance and medicals), 
private set intersection see 
\href{https://bit-ml.github.io/blog/post/bgv-fully-homomorphic-encryption-scheme-in-python/}{bgv repo}}

\subsection{The BGV Encryption Scheme}

While there are more encryption schemes than BGV (for instance BFV which is very similar,
 CKKS for floating point numbers and TFHE which I haven't encountered at all),
I have most of my experience with BGV, and it is a very nice scheme, 
where addition and multiplication associativity commutativity and more desired attributes 
hold.  Thus, we will focus on it.


BGV relies on the following parameters
\begin{itemize}
    \item $n$ (a power of 2), (the degree of $X^n+1$ from the RLWE problem).
    \item $q$ the ciphertext modulus.
    \item $t$ the plaintext modulus (usually plaintexts are "lifted" into the $R_q$ space)
    \item $\mathcal{X}$ the noise distribution, a discrete Gaussian.
\end{itemize}

In BGV we work with polynomials over $R_q=Z_q/(X^n+1)$.
This ring has polynomials with degree at most $n$ and integer coefficients in $(-q/2,q/2]$.

\begin{remark}
    It is important to understand the relationship between parameters, and choosing parameters 
    is not an easy task, and should not be tampered with lightly.
    There are some parameter generators online to find the best parameters
     for your cause see \href{https://github.com/Crypto-TII/fhegen}{fhegen}.
    
    Some things to consider are $n$, which affects the number of multiplications we can have,
    $q$ which, determines the safety of our scheme, and $t$ which determines 
    the size and efficiency of message packing into our plaintexts. 
    \emph{When $q$ reduces in size, it means better security}.    

    Notice that, smaller $q$ and smaller $n$ mean faster computations.
\end{remark}


\begin{remark}
    \begin{itemize}
        \item \red{check: I think we can treat coefficients as they were in $(0,q]$.}
        \item All polynomial operations are considered $\mod q$
         unless otherwise specified, to ease notation.
         \item When we talk about "small" polynomials, 
         we intuitively think about them as having small coefficients.
    \end{itemize}
\end{remark}

\paragraph{Plaintext and Ciphertext Spaces}
The plaintext and ciphertext spaces in BGV are defined over two distinct
polynomial rings. the plaintext space is denoted by the polynomial
ring $\mathcal{P}=R_{t}= Z_{t}\left[x\right]/\left(x^{n}+1\right)$,
that is, polynomials of degree less than $n$ with coefficients modulo
$t$. 
\begin{remark}
    Since Plaintexts must be polynomials in $R_q$, treat 
    $m$ as a string and encode each letter of the string as a coefficient of a polynomial
    $a_{0}a_{1}a_{2}\dots$
\end{remark}


the ciphertext space: $\mathcal{C}=R_q \times R_q$. 

\subsubsection{Algorithms}
In the following algorithms, $params$ is essentially the chosen security parameters. 
\paragraph{$sk,pk \leftarrow KeyGen(params)$}
The secret key $sk$ should be sampled from a distribution that outputs "small" elements 
with overwhelming probability. In practice (e.g., in Microsoft's SEAL library) $sk\in\{-1,0,1\}^n$,
with the probability of sampling 0 specified as a parameter, and the probability of sampling $-1$ or $1$
being equal. 

Once the secret key is derived, we can compute the public key $pk=(pk_0,pk_1)$.
First, draw a random polynomial $a\overset{r}{\leftarrow} R_q$ and 
noise (vector) $e\overset{r}{\leftarrow} \mathcal{X}$. Then compute 
$pk=(pk_0,pk_1)=(a\cdot sk+t\cdot e, -a)$.

Notice that here we rely on the hardness assumption? 
Can you guess why we multiply the noise by $t$?

\paragraph{$(c_0,c_1) \leftarrow encrypt(m,pk, params)$}
To encrypt we draw noise $e_0,e_1 \overset{r}{\leftarrow} \mathcal{X}$, and a "small" polynomial 
$u\in \{-1,0,1\}^n$ (like the secret key).
Then we compute the ciphertext as follows $$c=(c_0,c_1)= (pk_0\cdot u +t\cdot e_0+m, pk_1\cdot u +t\cdot e_1)$$

(Every element in the ciphertext tuple is in $R_q$).


\paragraph{$m \leftarrow decrypt(c,sk,params)$}

Decryption is performed by evaluating the ciphertext on the secret
key as follows $ m=((c_0+c1\cdot sk) \mod q) \mod t$

The following expansion of the expression above should make it clearer
\begin{align*}
c_{0}+c_{1}\cdot sk & =pk_{0}\cdot u+e_{0}\cdot t+m+(pk_{1}\cdot u+e_{1}\cdot t)\cdot sk=\\
 & =\underbrace{(a\cdot sk +e\cdot t)}_{pk_0} \cdot u + e_{0}\cdot t + m + (\underbrace{-a}_{pk_1}\cdot u+e_{1}\cdot t)\cdot sk=\\
 & = \cancel{a\cdot sk\cdot u} +e\cdot t\cdot u +e_{0}\cdot t + m  \cancel{-a\cdot u\cdot sk} +e_{1}\cdot t\cdot sk = \\
 & = e\cdot t\cdot u +e_{0}\cdot t + m  +e_{1}\cdot t\cdot sk = \\
 & = \underbrace{t\cdot (e\cdot u + e_0 +e_1\cdot sk)}_{error} + m
\end{align*}
If the error is small enough (doesn't cause $m$ to wrap around the modulus $q$), we should be able to eliminate the noise 
by applying $\mod t$ on the result above (e.g., remember that for every number $z$: $z\cdot t \equiv 0 \mod t$).

The decryption won't be correct when we have too much noise since $m$ parts of $m$
will be zeroed out by the $\mod q$ operation we perform.

\paragraph{addition and multiplication}
is simple, we treat the ciphertext tuples and the plaintexts as polynomials in a higher dimension,
that is $R_q[Y]$. The plaintexts are simply constant polynomials, and the freshly encrypted ciphertexts are polynomials 
of degree $1$.
It is a bit hard to see it in this manner, so let us discuss regular polynomials for instance.

Given $a(x)=a_0+a_1x$, $b(x)=b_0$, and $c(x)=c_0+c_1x$ adding them together or multiplying them is simple, right?
\begin{itemize}
    \item $a(x)+b(x)= (a_0+b_0)+a_1x$
    \item $a(x)\cdot b(x)= a_0\cdot b_0 + a_1\cdot b_0 x$
    \item $a(x) \cdot c(x)= (a_0\cdot c_0) +(a_0\cdot c_1 + a_1\cdot c_0)\cdot x + a_1\cdot c_1 \cdot x$
\end{itemize}

Okay, that's easy, right? Well, now you know how to perform addition and multiplication with plaintexts and ciphertexts!
$b(x)$ is a plaintext (since it has only one coefficient $b_0$), $a(x),c(x)$ are ciphertexts
since they are tuples.

In our ciphertext notions:
$p*c=(p\cdot c_0, p\cdot c_1)$,
$c*c'=(c_0\cdot c_0', c_0\cdot c_1' + c_1\cdot c_0', c_1\cdot c_1')$ and you can complete addition by yourself.

\subsubsection{Advanced Algorithms}
\paragraph{relianization}
The issue with multiplications is that they increase the noise very fast, and it means we'll need to work harder 
for decryption. That is, we increase the "degree" of our ciphertext every time we multiply it
Either we decrypt like so: $decrypt(c, sk, params)= c_0 +c_1\cdot sk+c_2\cdot sk^2+\dots$, 
or we perform relinearization.

I will not describe relinearization here but know that it can be performed between each multiplication
using something called an evaluation key (which is also public). After using the relianization algorithm 
a ciphertext of degree $3$ (i.e., $c\cdot c'$ is of degree $3$) for instance, will return to degree $2$.

\paragraph{key switch}
\red {TODO}
\paragraph{modulus switch}
is used to decrease the noise/error. once can "hop" to decreasing modulus values to reduce the noise, and thus allow 
more multiplications. \red{TODO}.

\paragraph{rotations}
\red{TODO}
\subsubsection{bootstrapping}
Read \href{https://www.zama.ai/post/what-is-bootstrapping-homomorphic-encryption}{bootstrapping for dummies} by Nigel Smart.

% \section{security of the scheme}

% Choosing optimal BFV parameters that maximize performance and respect
% security and functionality constraints is an art that is practiced
% by expert cryptographers. but there is a standard one can follow.

% But mainly, we first start by choosing the max size of integers $\left(q\right)$
% for the coefficients, and the max polynomial degree $\left(n\right)$.
% but generally, larger $n$ gives more security (but slower ops), larger
% $q$ means we can do more complex computations. 

% \paragraph{Ciphertexts in these encryption schemes contain a noise component
% (which is important for security), and that noise grows with each
% operation (The encrypted result can only be decrypted if the noise
% is smaller than $q$, hence using larger values of $q$ imply that
% we can do more operations).}

% \section{Relinearization}

% Remeber we generated another key along with the private and public.
% called $ek=\left(ek_{0},ek_{1}\right)$. using this key as randomness
% source, we can relinearize the result of multiplication: the new ciphertext
% would be :
% \begin{align*}
% c_{1} & =\left[C_{1}^{*}+ek_{0}\cdot C_{3}^{*}\right]_{q}\\
% c_{2} & =\left[C_{2}^{*}+ek_{1}\cdot C_{3}^{*}\right]_{q}
% \end{align*}
% Remember, decryption is : $c_{1}+c_{2}\cdot sk$. After doing that,
% the decryption would result with: 
% \[
% C_{1}^{*}+C_{2}^{*}\cdot sk+C_{3}^{*}\cdot sk^{2}+C_{3}^{*}\cdot e
% \]
%  which would decrypt just fine, but with a big error $\left(C_{3}^{*}e\right)$!
% We can fix that by using base decomposition.


% \section{Base Decomposition}
% \section{Mod Switching}

% \section{Choosing Parameters}



\chapter{Secret sharing, distributed key generation and threshold cryptography}
\section{Shamir secret sharing}
Given a secret $S$, generate random polynomial 
$p = S + a_0\cdot x + a_1\cdot x^2+a_2\cdot x^3+\dots$ of degree $f+1$.
create shares $(p(1),p(2),\dots, p(n))$ where $n\ge f+1$ and send it to nodes $(1,2,...n)$.
The nodes will publish their shares when they want to reconstruct the secret.


Once a party in the protocol has a hold of $f+1$ shares, 
it can recover the secret by creating an equations system,
and solve for the coefficients using $f+1$ shares as follows:
\begin{align*}
  & S +a_0\cdot 1^1 + a_1\cdot 1^2  + \dots = p(1)\\
  & S +a_0\cdot 2^1 + a_1\cdot 2^2  + \dots = p(2)\\
  & \vdots \\
  & S +a_0\cdot (f+1)^1 + a_1\cdot (f+1)^2  + \dots = p(f+1)
\end{align*}


This solution is simple, but is limiting in other factors (see~\ref{lagrange-in-exponent}).
In addition, it isn't very efficient; most known algorithms for solving linear equations 
demand $O(n^3)$.
One can improve performance by relying on Lagrange interpolation, which 
can be implemented in $O(n^2)$ (again, see \ref{lagrange-in-exponent}).


\paragraph{Improving Performance}
While Lagrange interpolation is highly efficient when interpolating, it assumes we don't have 
any structure over the points that the polynomial is evaluated on.

By relying on roots of unity, we can both create the shares of the polynomial, and reconstruct it 
via NTT (see~\ref{ntt}). 
Doing so, we can perform secret sharing or secret reconstruction in $O(n\log n)$ complexity.

\paragraph{Error Corrections}
\red{TODO}
\section{Verifiable secret sharing}
The dealer will generate a random polynomial $P(x)=S+a_1\cdot x +a_2\cdot x^2 +\dots a_n\cdot x^n$. 
It'll broadcast the polynomial as follows: $\{g^{a_i} | i\in [n]\cup {S}\}$.
Then it'll send to each node $i$ its share of the polynomial $P(i)$. 
To verify any share, including shares of other nodes, compute the following: 
$$ g^{P(i)} =
 g^{S}\cdot (g^{a_1})^{i} \cdot (g^{a_2})^{i^2}\cdots  =
 \prod_{j=0}^{k} (g^{a_j})^{i^j} = 
 g^{\sum_{j=0}^{t} a_j \cdot i^j} = 
 g^{P(i)} $$

Every node can compute for any $i$ the value $g^{P(i)}$ using the hidden coefficient broadcasted by the dealer.  
The ability to compute $g^{P(i)}$ guarantees that no one can publish a corrupt share and fool other nodes when they reconstruct the secret.


\section{Distributed key generation}
\subsection{protocol}
Assumptions, $n$ players, where each player has a broadcast channel (Might cost $n^2$ to publish through) and
a private channel to the other players.
Note: I'll use player/ member interchangeably.
\begin{enumerate}
  \item Each member performs VSS.
  That is, player $i$ generates a random polynomial 
  $p_{i}(x)=a_0+a_1\cdot x +a_2\cdot x^2 +\dots a_t\cdot x^t$, and broadcasts $ \{g^{a_i} | i \in [t]\}$.
  Then it sends a private share $ p_{i}(j)$ to each player $j$ where $j>0$.
  \item Each member verifies the share it receives according to the VSS protocol.
  \item Each member $j$ takes the shares it received $\{p_{1}(j),p_{2}(j),\dots\}$ and adds them together. 
  Then publish $ g^{y_i}=g^{\sum_{i}{p_{i}(j)}}$.
  \item Each member verifies these shares. How? They take all the broadcasted values from the first step
  as follows:$ \{g^{a_{ij}} | i \in [t], j \in Players\}$ and compute:
  $$\{\prod_{j}{ g^{a_{1j}}}, \prod_{j}{ g^{a_{2j}}},\dots\}=$$
  $$\{ g^{\sum_{j}{a_{1j}}}, g^{\sum_{j}{a_{2j}}}\} $$ 
  Each member should be able to verify the share their peers sent using these values.
  Denote $\mathcal{Q}$ the distributed polynomial.
  Example:$$g^{\mathcal{Q}(k)}=\prod_{i}(g^{\sum_{j}{a_{ij}}})^{k^i}= g^{\sum_{i}{p_i(k)}}=g^{y_k}$$
\end{enumerate}
\subsection{complexity}
\paragraph{Communication complexity} varies with the costs of broadcasting. using reliable broadcast means
Each of these broadcasts costs $O(n^2)$. with up to constant rounds of broadcasting,
 we get $O(n^3)$ total communication costs in the system. 
 Without reliable broadcast - $O(n^2)$ communication costs
 \paragraph{Computation complexity.} The top computation cost is creating $\mathcal{Q}$ then verifying each value.
 That is, computing $ \{ g^{\sum_{j}{a_{1j}}}, g^{\sum_{j}{a_{2j}}},\dots\} $  is $O(n^2)$, and computing it for each player: $O(n^3)$ 


\section{Largange interpolation in the exponent}\label{lagrange-in-exponent} %https://crypto.stackexchange.com/questions/67500/how-to-apply-lagrange-interpolation-on-bilinear-pairings
We can modify cryptographic schemes using Lagrange interpolation to work with $f<n$ participants.

Given a dataset of coordinate pairs $(x_j,y_j)$ on 
a polynomial $P(X)=\sum_{i=0}^{k}{a_i^{i}\cdot x^k}$ we can interpolate it as follows:
 $$P(X)=L(x)=\sum _{j=0}^{k}y_{j}\ell _{j}(x)$$
 where $\ell _{j}(x)$ is:

 $$\prod_{0\leq m\leq k, \ m\neq j}\frac {x-x_{m}}{x_{j}-x_{m}}$$
That is, we ensure $L(x_j)=y_j$.

\subsection{Using it in crypto}
For our convinicece, we publish onlt shares of the form: $\{(i,P(i))| i > 0\}$.

Denote: $i_j\in T\subseteq [n]$,  means index of player $j$.
in order to reconstruct we'll need enough indices such that $|T|=deg(P)+1$.

So to evaluate $L(0)$ we'll need to compute for each $i_j\in T$:
$$
 \ell_{i_j}(0) = \frac{
 \prod _{k\in T \setminus \{i_j\}}{\textbf{0}-k}
}{
  \prod_{k\in T \setminus \{i_j\}}{ (i_j - k)}
}
$$
Now, let us finally evaluate the secret:
$$ L(0) = \sum_{i_j \in T} {\ell_{i_j}\cdot P(i_j)} $$


\subsubsection*{$n-out-of-n$ protocol}
We'll take a complex and unfamiliar example to ensure one can apply it to other schemes.
NOTE: We rely in the following example on pairing-based cryptography. 
It'll have slight differences. 
for example, using the key pair $(sk, g^{sk})$ one can sign a message $m$ as follows: $H(m)^{sk}$.

Let's begin:


A node can publish a secret tied to a specific ID in the following protocol (Assume we have pairings).
Each user has a key pair $(sk_{i}, pk_{i})$, 
and the shared public key would be $\prod_{i}pk_{i}=pk$.
that is $\prod_{i}g^{sk_{i}}=g^{\sum_{i}sk_{i}}$.


to hide a new transaction $tx$: sample new keys: 
$\left(sk',pk'\right)$ and a random number $r$.
\begin{itemize}
\item compute $\gamma=Enc_{pk'}\left(tx\right)$
\item compute $c=\left(\underbrace{g^{r}}_{c_{L}},\underbrace{e\left(pk,H\left(\gamma\right)^{r}\right)\cdot sk'}_{c_{R}}\right)$ 
\item send $\left(c,\gamma,pk'\right)$.
\end{itemize}


Now, when every server signs: $\sigma_{i}=H\left(\gamma\right)^{sk_{i}}$\\
we can compute the product signature: $\sigma=\prod H\left(\gamma\right)^{sk_{i}}$
which can be used to decrypt the temp secret key: 
\begin{align*}
 & \frac{c_{R}}{e\left(c_{L},\sigma\right)}=\frac{e\left(g^{x},H\left(\gamma\right)^{r}\right)sk'}{e\left(g^{r},\sigma\right)}=\\
 & =\frac{e\left(g^{\sum sk_{i}},H\left(\gamma\right)^{r}\right)\cdot sk'}{e\left(g^{r},H\left(\gamma\right)^{\sum sk_{i}}\right)}=\\
 & =\frac{e\left(g^{\sum sk_{i}},H\left(\gamma\right)^{r}\right)\cdot sk'}{e\left(g,H\left(\gamma\right)\right)^{r\cdot\sum sk_{i}}}=\\
 & =\frac{e\left(g,H\left(\gamma\right)\right)^{r\sum sk_{i}}\cdot sk'}{e\left(g,H\left(\gamma\right)\right)^{r\cdot\sum sk_{i}}}=\\
 & =sk'
\end{align*}
 we managed to find out the temp secret key. now we can get the value
under $\gamma=Enc_{pk'}\left(tx\right)$.


\subsubsection*{$f-out-of-n$ protocol}
We assume all nodes have shares of the same polynomial $P$.
This protocol will have slight changes from the one described before.
It'll ensure we need only $f-out-of-n$ signatures to reconstruct the secret.


Each player holds a share $s_j$ (and a matching public key $g^{s_j}$), generated from Polynomial $P$. 
The shared secret key is $P(0)= S = \sum_{k \in T}{\ell_k(0)\cdot s_k}$,
and the public key is 
$$
  PK=
  \prod_{k\in T}{(g^{s_k})^{\ell_k(0)}} =
  g^{\sum_{k\in T}{\ell_k(0)\cdot s_k}} =
  g^{L(0)}=g^{P(0)}=g^S 
$$
for $|T|=f+1$ indices.

\paragraph*{Creating new secret}
For each transaction $tx$: sample new keys: $\left(sk'\right)$ and a random number $r$.
\begin{itemize}
\item compute $\gamma=Enc_{sk'}\left(tx\right)$
\item compute $c=\left(\underbrace{g^{r}}_{c_{L}},\underbrace{e\left(PK, H\left(\gamma\right)^{r}\right)\cdot sk'}_{c_{R}}\right)$ 
\item send $\left(c, \gamma\right)$.
\end{itemize}



\paragraph*{reconstruction}


With $f+1$ signatures over $H(\gamma)$, we can recompute the secret and extract out of $c$ the decryption key $sk'$.
When a player receives a signature $\sigma_{i}=H\left(\gamma\right)^{s_{i}}$
It verifies the signature and then stores $ (\sigma_i)^{\ell_i(0)}$.
With enough signatures, we can compute the product signature: $$
\sigma=\prod_{k\in T}{\sigma_k^{\ell_k(0)}} = 
\prod_{k\in T}{H\left(\gamma\right)^{\ell_k(0)\cdot s_{k}}} = 
H\left(\gamma\right)^{\sum_{k\in T}{\ell_k(0)\cdot s_{k}}}=
H\left(\gamma\right)^{L(0)} =
H\left(\gamma\right)^{S}
$$
Which can be used to decrypt the temp secret key: 
\begin{align*}
 & \frac{c_{R}}{e\left(c_{L},\sigma\right)}=\frac{e\left(PK, H\left(\gamma\right)^{r}\right)\cdot sk'}{e\left(g^{r},\sigma\right)}=\\
 & =\frac{e\left(g^S, H\left(\gamma\right)^{r}\right)\cdot sk'}{e\left(g^{r},H(\gamma)^S\right)}=\\
 & =\frac{e\left(g^S, H\left(\gamma\right)^{r}\right)}{e\left(g^{r},H(\gamma)^S\right)} \cdot sk'=\\
 & =\frac{e\left(g, H\left(\gamma\right)\right)^{S\cdot r}}{e\left(g,H(\gamma)\right)^{r\cdot S}} \cdot sk'=\\
 & =sk'
\end{align*}
 we managed to find out the temp secret key. now we can get the value
under $\gamma=Enc_{pk'}\left(tx\right)$. as wanted.

\chapter{protocols for identification and login}
\section{Identification Protocols}.

\subsection{Types of adversaries}

The identification protocol is a set of a prover $P$ and a verifier
$V$. where the prover wants to gain access to some resource, and
to do so, the verifier must give it access. the verifier will only
give access to a prover that convinces it that it has the rights to
that resource. for example, $P$ wants to enter its house, and to
do so it must unlock the door. the convincing method is a key.
\begin{description}
\item [{Direct$\,\,$attacks}] The adversary is not able to eavesdrop.
Then using no information other than what is publicly avail- able,
the adversary must somehow impersonate the prover to the verifier.
A simple password protocol is sufficient to defend against such direct
attacks.
\item [{Eavesdropping$\:$attacks}] The adversary can eavesdrop and obtain
the transcript of several interactions between the prover and verifier.
In this case the simple password protocol is insecure. However, a
slightly more sophisticated protocol based on one-time passwords (TBD)
is secure.
\item [{Active$\:$attacks}] an active adversary that interacts with the
prover. The adversary uses the interaction to try and learn something
that will let it later impersonate the prover to the verifier. Identification
protocols secure against such active attacks require interaction between
the prover and verifier. They use a technique called challenge-response.
\end{description}

\subsection{Salting}

\paragraph{Storing the password in plaintext is prone to active attacks, once
the verifier is compromised, the adversary can gain access easily.
A simple solution would be to hash the plain-texts. unfortunately
the space of possible passwords being used is not very large. Hence
in addition to hashing the plain-texts, add salt! }

\paragraph{To salt a plain-text, the verifier should add some random string
to it, then store the salt as plain-text, and hash the password and
the salt concat to it ($salt,Hash\left(salt||password\right)$).}

\paragraph{Salting ensures a direct attack would need to run an exhaustive search
over $D\times S$ where $D$ is the dictionary with all weak passwords
and $S$ is the space of random strings the verifier uses, for example
salt size can be $|S|=2^{128}$. }

\paragraph{That is not really the case though, it just ensures the attacker
cannot use a preprocessing algorithm to have efficient attacks. (?the
attacker needs to make a preprocess for all salts it saw in the server
along with the dictionary.) \protect \\
Now its best case would be to run on a dict along hash the plaintext
hash with the word from the dict in the hopes of finding a similar
hash in the servers password file.}

\subsubsection{salt and pepper}

skipped it

\subsection{eavesdrop safe}

\subsubsection{Hotp}

HOTP is a protocol for identification which is safe against eavesdrop
attacks. 

\paragraph{This attack attempts to change the password after each use. To do
so, the prover and verifier should share a secret, which they will
then be used to generate some proof of authenticity. Using $HMAC$
(or a $PRF$ as follows $F\left(k,i\right)$) the prover generates
a new value over a shared counter: $Hmac\left(i\right)$ and will
send it to the verifier. verifier can authenticate it. and then they
both advance $i$ by $1$. }
\begin{rem}
We can improve it by sending both $i$ and the output of the $HMAC$
over $i$, this way the verifier can check if $i$ is greater than
its counter $i'$, so they do not have to be matching counters at
all times.
\end{rem}

\begin{rem}
This verification protocol changes state only upon an attempt to login.
A user that does not login frequently, might be compromised if an
attacker gained the latest token. The attacker can sell the obtain
token to anyone which should give access once unless the user logins
again. 
\end{rem}


\subsubsection{TOPT: time base OTP }

Instead of a counter, one can use the current time, and each login
attempt is only valid if the timestamp is recent enough, say 10 seconds.

\subsubsection{S/key }

TODO.

attempts to solve the leaking key issue: \\
generate random $k$, give the verifier $H^{(n+1)}(k)$. the prover
stores $(k,n)$ to send identification request: 
\begin{itemize}
\item $P$ sends $t=H^{i}(k)$ then sets $i=i-1$. 
\item $V$ will inspect that $H(t)$ equals to what it stored, if it does
it updates the storage as $t$.
\end{itemize}
this can be used for $n$ times, afterwards needs to be regenerated.
kinda cool. 

basically, the provers sends $H^{n}(k)$ then $H^{n-1}\left(k\right)$
... $H(k)$.
\begin{itemize}
\item Assumes that the key is kept secret by the client
\item the resulting password is at least 128bit, which is long to type by
humans. 
\end{itemize}
%
You reached 18.6 in the book, which is talking about security against
active attacks. 

\section{Active attack}

the attacker can impersonate the verifier for a brief moment, and
its goal is to find a secret to fool the actual verifier. 

thus HOTP is easy to break, gain a legit one time password and forward
it to the verification protocol. 

\subsection{challenge-response protocol: }

the verifier sends a random challenge $c$, the prover uses a mac
over the challenge $c$ and send it back. the attacker can only interact
with the prover, it cannot prepare ask the prover all possible challenges
the verifier will chuck at him. and because we use $MAC$ the attacker
cannot forge a response from the prover. 

\subsubsection{improvement }

to ensure the verifier can be compromised and the attacker can't do
anything to gain access later using active attack - we use signature
scheme instead of mac. 

\chapter{Identification and signatures from sigma protocols}

This chapter will be useful to develop zero--knowledge proofs.

\section{Schnorr\textquoteright s identification protocol }

\paragraph{Schnorr's identification protocol is a basic block for building signature
schemes with DH assumption as its base.}

This will be our first proof of knowledge ever to be seen.

\paragraph{This protocol can be proved secure against eavesdropping attacks,
assuming the discrete logarithm problem is hard.}

Let $\G$ be a cyclic group of prime order $q$with a generator $g\in\G$.
Suppose the prover has a secret key $\alpha\in\Z_{q}$ and a corresponding
public key $u=g^{\alpha}$. $P$ wishes to prove to an identifier
$V$ that it knows $\alpha$.

\subsubsection{protocol}

both the prover and the verifier generates random numbers respectively
$k,c$. The prover send $g^{k}$ and once it receives the challenge
$c$ from the verifier, it'll need to compute $t=k+\alpha\cdot c$
and send $t$ to the verifier. in turn, the verifier should compute
the following:$\left(u\right)^{c}\cdot g^{k}=g^{k+\alpha\cdot c}$
which should be equal to $g^{t}$.

\paragraph{A keen eye could see that this is very similar already to Schnorr's
signature scheme.}

\subsubsection{Honest verifier zero knowledge and security against eavesdropping}

We prover that the protocol is eavesdrop safe by assuming the adversary
has access to $vk$, and that it had seen a conversation between $P$
and $V$. The idea is to show that these conversations do not help
the adversary, because the adversary could have efficiently generated
these conversations by himself, given $vk$ (but not $sk$).
\begin{defn}
Let $\mathcal{I}=(G,P,V)$ be an identification protocol. We say that
$\mathcal{I}$ is honest verifier zero knowledge, or $HVZK$ for short,
if there exists an efficient probabilistic algorithm $Sim$ (called
a simulator) such that for all possible outputs $(vk,sk)$ of $G$,
the output distribution of $Sim$ on input $vk$ is identical to the
distribution of a transcript of a conversation between $P$ (on input
$sk$) and V (on input $vk$).
\end{defn}


\paragraph{Comments on the terminology are in order. The term \textquotedblleft zero
knowledge\textquotedblright{} is meant to suggest that an adversary
learns nothing from $P$, because an adversary can simulate conversations
on his own (using the algorithm $Sim$), without knowing $sk$. The
term \textquotedblleft honest verifier\textquotedblright{} conveys
the fact this simulation only works for conversations between $P$
and the actual, \textquotedblleft honest\textquotedblright{} verifier
$V$ , and not some arbitrary, \textquotedblleft dishonest\textquotedblright{}
verifier, such as may arise in an \textbf{active attack} on the identification.}
\begin{thm}
If an identification protocol $\mathcal{I}$ is secure against direct
attacks, and is HVZK, then it is secure against eavesdropping attacks. 
\end{thm}
The main idea here: if I'm secure against someone interacting maliciously with the verifier.
and in addition to that, my regular protocol does not provide any useful information someone
can use later on: im safe against eavsedrop attacks too.

Schnorr's protocol is HVZK, they prove it using a simulator.
In addition to that, they show it's safe against direct attacks by showing that 
if there was some attacker that can break it, it can generate twice accepting answers
to the verifier.
then they can use these two answers to BREAK the discrete log assumption.


\subsection{ECDSA }

The ECDSA signature scheme $(G,S,V)$ uses the group of points $\G$
of an elliptic curve over a finite field $F_{p}$. Let $g$ be a generator
of $\G$ and let $q$ be the order of the group $\G$, which we assume
is prime. Im skipping the algorithm, because it isn't as elegant.
in addition to that, Schnorr sigs are strongly secure, but ECDSA are
not. Given an ECDSA signature $\sigma=\left(r,s\right)$ on a message
$m$, anyone can generate more signatures on $m$.


\section{Sigma protocols: basic definitions }

Schnorr\textquoteright s identification protocol is a special case
of an incredibly useful class of protocols called Sigma protocols.
We'll soon see the basic concepts associated with Sigma protocols
and how they can become helpful to us.\\
Later we'll see how Sigma protocols are useful to us, even out of
the context of identification and signatures.
\begin{defn}
(Effective relation). An effective relation is a binary relation $R\subseteq X\times Y$,
where $X,Y$ and $R$ are efficiently recognizable finite sets. Elements
of $Y$ are called \textbf{statements}. If $\left(x,y\right)\in R$,
then $x$ is called a witness for $y$.
\end{defn}

\begin{defn}
Sigma Protocol. Let $R\subseteq X\times Y$ be an effective relation.
A \textbf{Sigma Protocol }for $R$ is a pair $\left(P,V\right)$.
\begin{itemize}
\item $P$ is an interactive protocol algorithm called the \textbf{prover},
which takes as input a witness--statement pair $\left(x,y\right)\in R$.
\item $V$ is an interactive protocol algorithm called the \textbf{verifier},
which takes as input a statement $y\in Y$, and which outputs accept
or reject.
\item $P$ and $V$ are structured so that an interaction between them always
works as follows:
\begin{itemize}
\item To start the protocol, $P$ computes a message $t$, called the commitment,
and sends $t$ to $V$;
\item Upon receiving $P$\textquoteright s commitment $t$, $V$ chooses
a challenge $c$ at random from a finite challenge space $C$, and
sends $c$ to $P$;
\item Upon receiving $V$ \textquoteright s challenge $c$, $P$ computes
a response $z$, and sends $z$ to $V$;
\item Upon receiving $P$\textquoteright s response $z$, $V$ outputs either
\textbf{accept} or \textbf{reject}, which must be computed strictly
as a function of the statement $y$ and the conversation $\left(t,c,z\right)$.
In particular, $V$ does not make any random choices other than the
selection of the challenge --- all other computations are completely
deterministic.
\end{itemize}
\end{itemize}
We require that for all $\left(x,y\right)\in R$, when $P\left(x,y\right)$
and $V\left(y\right)$ interact with each other, $V\left(y\right)$
always outputs accept.
\end{defn}

\begin{rem}
So a sigma protocol, is any prover verifier challenge where the prover
knows some witness to y and want to prove that $y$ is in the language
to the verifier?
\end{rem}

If the output is accept we call the conversation $\left(t,c,z\right)$
an accepting conversation for $y$.
\begin{example}
It should be clear that for Schnorr\textquoteright s identification
protocol $\left(G,P,V\right)$, the pair $\left(P,V\right)$ is an
example of a \textbf{Sigma protocol} for the relation $R\subseteq X\times Y$,
where
\[
X=\Z_{q},Y=\G\text{ and }R=\left\{ \left(\alpha,g^{\alpha}\right)\in\Z\times\G\right\} .
\]

TODO continue
\end{example}


\chapter{MPC}
We have a function f that takes n inputs and produces m outputs:
$$  (y_1,...,y_m) = f(x_1,...,x_n).$$
We also have N parties, $P_1,...,P_N$. 
Their goal is to run a protocol where each input value $x_i$ is 
contributed by one of the parties and each output $y_j$ is obtained by one or more of
 the parties.


We want to design an efficient protocol for this problem that provides
\emph{ privacy, soundness, and input independence}.

\section{Wanted properties}
\begin{itemize}
  \item Privacy: No part learns anything about any other part's inputs
  (except for the information that is inherently revealed by the outputs).
  \item Soundness/correctness: honest parties compute correct outputs. 
  \item Input independence: All parties must choose their 
  inputs independently of the other parties' inputs 
  (No one can use Alice's inputs to their advantage).
  \item Guaranteed output delivery: all honest parties are guaranteed to obtain
  their output.
  \item Fairness: If any party (including corrupted members) obtains their output, 
  then all honest parties do so.
\end{itemize}
\section{Formal Security}
The way to define security formally is to say that an attack on the protocol in 
the "real world" 
is equivalent to some attack on the protocol in an "ideal world" in which 
no damage can be done.

In the ideal world, the protocol is implemented using a trusted 
party to which all parties (both honest and corrupt) submit inputs, 
and from which all parties (both honest and corrupt) obtain their designated outputs. 
The trusted party itself cannot be corrupted.

We then describe the details of the ideal-world execution, including the 
behavior of the trusted party who evaluates the function.

The definition of security then basically says: for every efficient adversary $A$ 
in the real world, there exists an "equivalent" efficient 
adversary $\mathcal{S}$ in the ideal world (usually called a simulator).
Since there is no possible attack in the ideal world, 
there is no possible attack in the real world.

the informal notions of \emph{privacy, soundness, and input independence} are
 implied by this type of definition.

 Essentially we'll want to show that the ideal world we describe is 
 "equivalent" to the real world.

 \subsection{A few applications of MPC}
 \begin{itemize}
  \item Elections: Without MPC, parties submit votes to a trusted server $T$ who 
  counts the votes and states the winner.
  \item Privacy-preserving mining of medical data. Two hospitals want to share data,
  to perform research, but cannot share it due to regulations.

 \end{itemize}

\section{Securely evaluating arithmetic circuits}

