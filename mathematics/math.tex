\part{Mathematics}
This part is a collection of notes on mathematics.
It includes notes on algebraic structures, number theory, and probability. 


\part{Algebraic structures}
This section was motivated by a theorem for polynomial identity, which is used in many cryptography applications:
\begin{lemma}[DeMillo-Lipton-Schwartz-Zippel lemma] 
\label{schawtz-zippel}
Let $p\in \F[x_1,x_2,x_3,\dots, x_n]$ be a non-zero polynomial of a total 
degree $d\ge0$ over a field $\F$. Let $S$ be a finite set of $\F$ and let 
$(r_1,r_2,\dots,r_n)$ be selected at random, \emph{independently and uniformly} from $S$.
Then 
$$ \Pr{[p(r_1,r_2,\dots,r_n)=0]}\le \frac{d}{|S|}$$
\end{lemma}

Unfortunately, this doesn't work over quotient polynomial rings (e.g., instead of
 $\F[x]$, we have $\F[x]/(x^n+1)$).

We'll start learning Algebraic structures in hopes we gain enough insight regarding \ref{schawtz-zippel}.
\chapter{Group theory}

\section{Group}
\begin{defn}
    A \emph{group} is a set $G$ with a binary operation $\cdot :G\times G\to G$ that satisfies the following constraints:
    \begin{itemize}
        \item the operation holds closure: if $x,y\in G$ so is $x\cdot y\in G$.
        \item the operation is associative: $\forall x,y,z\in G$ it holds that $(x\cdot y)\cdot z = x\cdot (y\cdot z)$.
        \item has an identity element: there exists $e\in G$ such that $\forall x\in G x\cdot e = x = e\cdot x$
        \item Every element $x\in G$ has an inverse $x^{-1}\in G$ such that $x\cdot x^{-1} = e = x^{-1} \cdot x$.
    \end{itemize}
\end{defn}

\begin{remark}
    Note that a group can be an additive group or a multiplicative group.
\end{remark}
\begin{example}
    Assume $\F*=F\\{0}$, $(\F*, \cdot, 1)$ is a valid multiplicative group but,
    $(\F, \cdot, 1)$ is not because there is no inverse to $0$ in $\F$.
\end{example}

\begin{defn}
    group $G$ is called \emph{commutative}, or \emph{abelian} if for every $x,y\in G$ $x\cdot y=y\cdot x$ holds.
\end{defn}

\subsection{traits of groups}
let $(G,\cdot,e)$ be a group, thus
\begin{itemize}
    \item $e$ is unique (like in fields).
    \item the inverse of an element is unique.
    \item $\forall x\in G (x^{-1})^{-1}=x$.
    \item $\forall x,y\in G (x\cdot y)^{-1}=y^{-1}\cdot x^{-1}$
\end{itemize}


\begin{defn}
    denote $|G|$ as the \emph{order} of a group - its size.
\end{defn}

\begin{defn}
    denote the order of $x\in G$ to be the minimal $m!=0\in\N$ such that  $x^m=e$. if there is no such $m$, then
    the order of $x$ is infinite.
\end{defn}

\begin{thm}
    Let there a finite group $G$, thus $\forall x\in G$ has a finite order.
\end{thm}

\begin{thm}
    Let $G$ be a group, and let $x\in G$ such that $order(x)=n$, thus $x^m=e$ iff $n|m$.
\end{thm}

\begin{thm}[Fermat's little theorem]\label{fermats-little-theorem}
    let there be $x$ and $p$ prime, thus $x^p \equiv x \mod p$
    (and as a result $x^{p-1}\equiv e mod p$).
\end{thm}

\subsection{Subgroups}

\begin{defn}
    Let there be a group $G$, $\emptyset \neq H\subseteq G$  is a \emph{subgroup} if it is a group
     with respect to the same operation of $G$.
\end{defn}
usually, we denote a subgroup with $H\le G$.

\subsection{cosets}



% \section{Motivation}
% \section{Ring theory intro}
% In this chapter, we'll discuss definitions, and theorems (many without proofs) extracted and summed by me from~\cite{galois-book-edinburgh} And HUJI's Algebraic structures 1.


% Historically, the subject of Galois theory was motivated by the desire to solve polynomial equations $p(x)=0$
% where $p(x)=\sum_{i=0}^{n}{a_i\cdot x^{i}}$.
% How do we solve for $x$ with some given degree $d$? we know how to do it for $d\in \{0,1,2,3,4\}$. 
% There was no formal way to solve for $x$ when $d\ge 0$. and it remained an open question for hundred of years.
% Unfortunatly,Raffini and Abel(1823) proved that there is no formula for roots of a quintic using only. $+, -, \times , \div, \sqrt[n]{}$.


% Evariste Galois gave a far more conceptual proof, inventing group theory on the way. 
% He died at 20 in a duel (1831). The paper was only published in 1846.

% \section{Rings and fields}

% \begin{defn}
%     We call a group $R$ a \emph{ring} if it is an abelian group (commutative) whose operation $+$,
%     is called addition, with a second operation $\cdot$ callend multiplication that is 
%     associative, distributive and has a multiplicative identity element.
% \end{defn}

% although it isn't always the case, we will assume all our rings are 
% commutative with a multiplicative identity.

% In terms of fields, a field is always a ring, 
% it assumes more properties on the ggiven group.
% a ring can have subsets which are fields.


% \begin{example}
%     There are many ways of building new rings from old. One of the most fundamental
%      is that from any ring $R$, we can build the ring $R[t]$ of polynomials over
%      $R$ (i'll define it formally later).
% \end{example}

% \begin{defn}
%     Given two rings, $R$ and $S$, a \emph{homomorphism} from $R$ to $S$ is 
%     a function $\phi : R\to S$ satisfying the following equations
    
%     \begin{align*}
%         \phi(x+y) = \phi(x) + \phi(y), && \phi(0)=0  \\
%         \phi(x\cdot y) = \phi(x)\cdot \phi(y), && \phi(1)=1 \textbf{\red{important!}}
%     \end{align*}
% \end{defn}
% Note that we gain $\phi(-x) = -\phi(x)$ too. furthermore, that we switch the multiplicity
% and addition operation to be the one that used in $S$
% that is $\phi(x) \cdot \phi(y)$ elements of $S$, and $x\cdot y$ are elements of $R$.

% lalalala
% \begin{lemma}
%      If for some function $\phi:R\to S$ it holds that: 
%     $\phi(x+y) = \phi(x) + \phi(y)$  and $\phi(x\cdot y) = \phi(x)\cdot \phi(y)$,
%     for all $x,y\in R$ than, $\phi$ is a homomorphism.
% \end{lemma}

% \begin{defn}
%     A sub ring of a ring $R$ is a subset $S \subseteq R$ that contains 0 and \red{1} 
%     and is closed under addition, multiplication and negatives.    
% \end{defn}

% \begin{defn}
%     If $R$ is a ring, we'll denote $R[X]$ the ring of polynomials
%     in $x$ with coefficients in $R$. Formally, an element of $R[X]$ is a finite
%     sum: $$ r_0 + r_1x +r_2x^2 +\dots +r_dx^d$$
%     With $d\in\Z\ge0$.
%     We treat $x$s as symbols, rather than actual values or variables.
%     \red{todo: define poly addition and multiplication}
% \end{defn}

% \begin{remark}[technical but important]
%     a polynomial $f(x)\in R[X]$ defined as $f(x)=\sum_{i=0}^{d}{a_i\cdot x^{i}}$. 
%     can be evaluated as a function over $R$, defining $f:R\to R$ by $f(r)=\sum_{i=0}^{d}{a_i\cdot r^{i}}$
%     However, in this generality, different polynomials can give rise to the same function!
%     \begin{example}
%         $R=\Z \\ 2\Z = {0,1}$ the two different polynomials
%         $f(x)=x, g(x)=x^2$ give rise to the same function $R\to R$.
%     \end{example}
% \end{remark}

% \section{Commutative Algebra}
% In this section we gradually define preliminaries for the schwarts-zippel lemma, 
% and polynomial homomorphism.

% \begin{defn}
%     Let $R$ be a ring. An element $a\in R$ is a \emph{unit} if there exists
%      $b\in R$ such that $a\cdot b = b\cdot a = 1$. The set of all units is denoted by $R^*$.
% \end{defn}

% \red{define zero divisors as well when you have the time}.

% \begin{defn}
%     Let $\alpha_0,\alpha_1,\dots \alpha_m-1\in R$. We call $A=\{\alpha_0,\alpha_1,\dots \alpha_m-1\}$ an exceptional set
%     if and only if $\alpha_i- \alpha_j\in R^*$ for all $i,j\in[m]$ with $i\neq j$. 
%     We define Lenstra constant of $R$ to be the size of the biggest exceptional subset of $R$.
% \end{defn}

% \begin{remark}
%     an exceptional set is a private case of a regular difference set,
%     I will not discuss this here.
% \end{remark}

% Now That we've defined exceptional sets, we can talk about the polynomial evaluation homomorphism:

% \begin{defn}[polynomial evaluation homomorphism]
%     \red{TODO}
% \end{defn}

% Given an exceptional set, one can prove  the following:
% \begin{lemma}[schwarts-zippel over rings~\cite{schwarts-zippel-over-rings}]
%     Let $R$ be a commutative ring and $f: R^n \to R$ be an $n$-variate non-zero 
%     polynomial. Let $A \subseteq R$ be an exceptional set. Then
%     $$ \Pr_{x\leftarrow A^n}{[f(x)=0]} \leq \frac{\text{deg} f}{|A|}$$
% \end{lemma}


\chapter{Probability}
\todo{take data from mathematical-tools course and put here.}

\section{Tricks and Techniques}

\paragraph{$\Pr{[A\cap B]}\ge ?$.}
Computing the exact probability of $A\cap B$ can be difficult, but we can often find a lower bound
by relying on the pidgenhole principle.

To compute it thus, we can write down the whole space, and plot it as a 
horizontal line starting at $0$ and ending at $1$.
then we plot a line to represent $A$. Given $A$ which starts from $0$ and ends at $a$,
we define $B$'s line as starting $1$ and ending at $b$ (see \autoref{probabilty:tricks:intersection}).

\begin{figure}
    % \centering
    \includegraphics[scale=0.75]{mathematics/illustrations/AcapB.pdf}
\end{figure}\label{probabilty:tricks:intersection}

For example, $\Pr[A]=1-\frac{1}{16}, \Pr[B]=\frac{3}{4}$, thus, $A\cap B \ge \frac{3}{4}-\frac{1}{16}$. That is, 
We take the size of $B$ and remove $1-\Pr[A]$ from it.


\input{mathematics/Lattice.tex}

% TODO: put inside the lattice file
\chapter{Lattices}
I think of lattices like discrete linear algebra. 
It doesn't have to be integers. for instance, $\sqrt{2}\cdot \mathbb{Z}^2$, is a valid lattice.
\todo{take the following and grab what you need from it: https://advancedcrypto.github.io/Lecture2.pdf}

Throughout, we treat all vectors as column vectors unless otherwise specified.

\begin{definition}[lattices]
    Given $n$ \bf{linearly independent} vectors $b_1,\dots, b_n\in \mathbb{R}^m$, the lattice generated by them is defined as
$$ 
\mathcal{L}(b_1,\dots, b_n) = \{\sum{x_i \cdot b_i} | x_i \in \mathbb{Z} \}
$$
    
and we call $b_1,\dots, b_n$ its \emph{basis}.
\end{definition}


\begin{definition}
We say that $n$ (the number of vectors in the basis) is the \emph{rank} of the lattice, and $m$ (from $\mathbb{R}^m$) is the \emph{dimension} of the lattice.
In addition, if $n=m$ it is considered \emph{full rank}.
\end{definition}

We discuss mainly full-rank lattices.


We will use a notational short-hand when dealing with bases, letting a matrix $B$ whose columns are $b_1,\dots, b_n$ the basis vectors denote a lattice basis. That is, we will write

$$ \mathbf{B}=\left(\begin{matrix}| &  & |\\
\mathbf{b_{1}} & \dots & \mathbf{b_{n}}\\
| &  & |
\end{matrix}\right) 
$$

\section{Same Lattice, Many Bases}

The same lattice can have many different bases (as in classical linear algebra)
For example, it turns out that all the bases given below generate the same 
lattice

$$ 
B_{1}=\left(\begin{matrix}1 & 0\\
0 & 1
\end{matrix}\right)\text{ and }B_{2}=\left(\begin{matrix}2 & 1\\
1 & 1
\end{matrix}\right)
$$


A natural question to ask is: how can we efficiently tell if two given bases $B$ and $b'$ generate the same lattice?
We will give two answers to this question - an algebraic answer and a geometric answer.

\subsection{An Algebraic and Geometric Characterization using Unimodular Matrices}
Our first characterization provides an efficient algorithm to determine if two bases generate the same
lattice. In order to present the characterization, we first need to define the notion of a unimodular matrix.


\begin{definition}
    A matrix $U\in \mathbb{Z}^{n\times n}$ is \emph{unimodular} if $| \det{U} |=1$ (the absolute value of the determinante of $U$).
\end{definition}

for instance, $\left(\begin{matrix}2 & 1\\
1 & 1
\end{matrix}\right)$ is unimodular, but $\left(\begin{matrix}42 & 41\\
    9 & 8
\end{matrix}\right) $.

\begin{proposition}
    If $U$ is unimodular, so is $U^-1$.
\end{proposition}

We can now state the characterization of equivalent bases.

\begin{theorem}
Given two full-rank bases $B,B'\in \mathbb{R}^{n\times n}$, the following two conditions are equivalent 
    \begin{itemize}
        \item $\mathcal{L}(B)=\mathcal{L}(B')$
        \item There exists a unimodular matrix $U$ such that $B=B'U$.
    \end{itemize}

    (Without proof)
\end{theorem}


\paragraph{A Geometric Characterization using the Fundamental Parallelepiped}
