
\chapter{Algebraic structures}
In this section I'll discuss basic definitions and theorems used in cryptography.
This is by no means a comprehensive guide, 
but rather a collection of definitions and theorems that I found useful.
% This section was motivated by a theorem for polynomial identity, which is used in many cryptography applications:
% \begin{lemma}[DeMillo-Lipton-Schwartz-Zippel lemma] 
% \label{schawtz-zippel}
% Let $p\in \F[x_1,x_2,x_3,\dots, x_n]$ be a non-zero polynomial of a total 
% degree $d\ge0$ over a field $\F$. Let $S$ be a finite set of $\F$ and let 
% $(r_1,r_2,\dots,r_n)$ be selected at random, \emph{independently and uniformly} from $S$.
% Then 
% $$ \Pr{[p(r_1,r_2,\dots,r_n)=0]}\le \frac{d}{|S|}$$
% \end{lemma}

% Unfortunately, this doesn't work over quotient polynomial rings (e.g., instead of
%  $\F[x]$, we have $\F[x]/(x^n+1)$).

% We'll start learning Algebraic structures in hopes we gain enough insight regarding \ref{schawtz-zippel}.
\chapter{Group theory}

\section{Groups Rings and Fields}
\begin{definition}
    A \emph{group} is a tuple $(G,\cdot,e)$ where $G$ is a set and $\cdot$ is a binary operation on $G$
    that satisfies the following constraints:
    \begin{itemize}
        \item the operation holds closure: if $x,y\in G$ so is $x\cdot y\in G$.
        \item the operation is associative: $\forall x,y,z\in G$ it holds that $(x\cdot y)\cdot z = x\cdot (y\cdot z)$.
        \item has an identity element: there exists $e\in G$ such that $\forall x\in G x\cdot e = x = e\cdot x$
        \item Every element $x\in G$ has an inverse $x^{-1}\in G$ such that $x\cdot x^{-1} = e = x^{-1} \cdot x$.
    \end{itemize}
\end{definition}


\begin{remark}
    Note that a group can be an additive group or a multiplicative group.
\end{remark}
\begin{example}
    Assume $\F*=F\\{0}$, $(\F*, \cdot, 1)$ is a valid multiplicative group but,
    $(\F, \cdot, 1)$ is not because there is no inverse to $0$ in $\F$.
\end{example}

\begin{definition}
    group $G$ is called \emph{commutative}, or \emph{abelian} if for every $x,y\in G$ $x\cdot y=y\cdot x$ holds.
\end{definition}


\begin{definition}
    A \emph{monoid} is a group without the requirement of inverses.
    That is, a monoid is a tuple $(M,\cdot,e)$ where $M$ is a set and $\cdot$ is a binary operation on $M$
    that satisfies the following constraints:

    \begin{itemize}
    \item the operation holds closure: if $x,y\in G$ so is $x\cdot y\in G$.
    \item the operation is associative: $\forall x,y,z\in G$ it holds that $(x\cdot y)\cdot z = x\cdot (y\cdot z)$.
    \item has an identity element: there exists $e\in G$ such that $\forall x\in G x\cdot e = x = e\cdot x$
    \end{itemize}
\end{definition}


Now that we've defined what a group is, let's highlight some important theorems and definitions:

\begin{definition}
    denote $|G|$ as the \emph{order} of a group - its size.
\end{definition}

\begin{definition}
    denote the order of $x\in G$ to be the minimal $m!=0\in\N$ such that  $x^m=e$. if there is no such $m$, then
    the order of $x$ is infinite.
\end{definition}

\begin{thm}
    Let there a finite group $G$, thus $\forall x\in G$ has a finite order.
\end{thm}

\begin{thm}
    Let $G$ be a group, and let $x\in G$ such that $order(x)=n$, thus $x^m=e$ iff $n|m$.
\end{thm}

From a group, we can define a field, which is a more complex structure.

\begin{definition}
    A \emph{field} is a tuple $(F,+,\cdot,0,1)$ where $F$ is a set and $+,\cdot$ are binary operations on $F$
    such that $(F,+,0)$ is an abelian group, $(F,\cdot,1)$ is an abelian group, and the distributive law holds.

    \begin{align*}
        \forall x,y,z\in F, && x\cdot(y+z) = x\cdot y + x\cdot z
    \end{align*}
\end{definition}

\begin{example}
    The set of real numbers $\R$ is a field. 
    The set of integers $\Z$ is not a field.
    And every set of integers modulo a prime $p$ is a field.
\end{example}

With the above example, we remind the reader of Fermat's little theorem, which 
is used in many theorems or applications of group theory. In this summary, I rely on Fermat's little theorem
when discussing NTT and FFT (see \autoref{section:fft}).


Some use the notation $GF(q)$ to denote a finite field of size $q$.
\begin{thm}
    Let $q$ be a prime, thus $GF(q)$ is a field.
\end{thm}

\begin{thm}[Fermat's little theorem]\label{fermats-little-theorem}
    let there be $x$ and $p$ prime, thus $x^p \equiv x \mod p$
    (and as a result $x^{p-1}\equiv e mod p$).
\end{thm}


Now that we defined groups and fields, we can define rings.
\begin{definition}
    A \emph{ring} is a tuple $(R,+,\cdot,0,1)$ where $R$ is a set and $+,\cdot$ are binary operations on $R$
    such that $(R,+,0)$ is an abelian group, $(R,\cdot,1)$ is a monoid, and the distributive law holds.

\end{definition}

In other words, a ring is a group with respect to addition, and a monoid with respect to multiplication.

\begin{example}
    The set of integers $\Z$ is a ring, since some numbers don't 
    have an inverse (e.g., $2$ doesn't have an inverse in $\Z$). 
\end{example}


\subsection{Polynomial Rings}
Polynomials are heavily used in cryptography, and are a fundamental concept in algebra.
\begin{definition}
    Let $R$ be a ring, we'll denote $R[X]$ the ring of polynomials
    in $x$ with coefficients in $R$. Formally, an element of $R[X]$ is a finite
    sum: $$ r_0 + r_1x +r_2x^2 +\dots +r_dx^d$$
    With $d\in\Z\ge0$.

    We treat $x$s as symbols, rather than actual values or variables.

    We define polynomial addition and multiplication as follows:
    \begin{align*}
        f(x) + g(x) &= \sum_{i=0}^{d}{(a_i+b_i)x^i} \\
        f(x) \cdot g(x) &= \sum_{i=0}^{d}{\sum_{j=0}^{d}{a_i\cdot b_j\cdot x^{i+j}}}
    \end{align*}
\end{definition} \label{def:polynomial-ring}\label{def:polynomial}

Since we discuss formal mathematics, it is important to remember that while polynomials seem like functions,
in our context they are just a formal sum of terms and do not represent functions.
As a result, to evaluate a polynomial, we'll need to define a function that receives a polynomial 
and a value and returns the result of the polynomial at that value.


\begin{definition}
    Ideal $I$ is a subset of a ring $R$ that is closed under addition and, or multiplication by elements of $R$.

    Furthermore the product of an element of $R$ and an element of $I$ is in $I$.
    That is, $\forall r\in R$ and $i\in I$ it holds that $r\cdot i\in I$.
\end{definition}

For instance, the group $7\Z=\{7\cdot i:i\in\Z\}$ is an ideal of $\Z$ because it is closed
under addition and multiplication by elements of $\Z$,
and the product of an element of $\Z$ and an element of $7\Z$ is also in $7\Z$.

Given an ideal $I$ and a ring $R$, we can define the quotient ring $R/I$.
This ring is defined as the set of equivalence classes of $R$ under the equivalence relation 
$x\sim y$ iff $x-y\in I$ (that is, there exists $i\in I$ such that $x-y=i$).
This $\sim$ notation can be confusing, so we'll denote the equivalence class of $a$ as $[a]=\{b\in \Z :a\sim b \}$.
Furthermore, we note that $[a]+[b]=[a+b]$ and $[a]\cdot[b]=[a\cdot b]$.

\begin{example}
    $\Z/7\Z$ is the set of equivalence classes of $\Z$ under the equivalence relation
    $x\sim y$ iff $x-y\in 7\Z$. 
    For instance, $[0]=7\Z$, $[1]=1+7\Z$, $[2]=2+7\Z$, and so on.

    One can quickly see that there are only 7 equivalence classes, because $[7]=7+7\Z=0+7\Z$ (we wrap around, 
    and don't care about the actual value of the number, only its residue modulo 7).

    This specific quotient ring is also known as $\Z_7$, which is also a field.
\end{example}


\begin{bclogo}[logo=\bcinfo, couleurBarre=orange, noborder=true, couleur=white]{How I think about ideals}
    This formal definition is quite confusing. I think about these ideals as groups 
    with values we don't care about, and $R/I$ defines $R$ without 
    the values we don't care about.

    So in the case of $Z/7Z$, we don't care about 
    all the multiplications of $7$, only the first 7 of them.
\end{bclogo}


Another important example if the polynomial quotient ring $R[X]/X^d$. 
This ring is the set of equivalence classes of $R[X]$ under the equivalence relation
$f\sim g$ iff $f-g\in X^d$. Or in other words, all polynomials of degree less than $d$ are equivalent.


\section{Extension Fields}

One can take a field $GF(q)$, and extend it.
To do so, first find a polynomial $p(x)$ of degree $d$ that is irreducible over $GF(q)$
(We'll use prime polynomials).
Then, define the extension field $GF(q^d)$ as the set of polynomials of degree less than $d$ with coefficients in $GF(q)$.
THis field is basically $GF(q)[X]/p(X)$. BUT, Once we have this field, 
we can define it in order, and change the notations of the polynomials to be integers 
in the range $[0,q^d-1]$. This is a way to extend the field, and get a larger field.


For instance $GF(2)$ can be extended to any power $m$ of 2.

For instance, $GF(2)[X]/(X^2+X+1)$ is $GF(2^2)$, and has the following elements:
$0,1,X,X+1$. Once we define the elements and how they interact with their operations, we can 
switch their notation to be integers in the range $[0,3]$. 


\section{Tricks}
\todo{If you have any tricks or theorems you think are important, put them here.}
% TODO: discuss Ittai's trick for $A\cap B$ probabiltiy.