
\chapter{Algebraic Structures}
In this section I'll discuss basic definitions and theorems used in cryptography.
This is by no means a comprehensive guide, 
but rather a collection of definitions and theorems that I found useful.
% This section was motivated by a theorem for polynomial identity, which is used in many cryptography applications:
% \begin{lemma}[DeMillo-Lipton-Schwartz-Zippel lemma] 
% \label{schawtz-zippel}
% Let $p\in \F[x_1,x_2,x_3,\dots, x_n]$ be a non-zero polynomial of a total 
% degree $d\ge0$ over a field $\F$. Let $S$ be a finite set of $\F$ and let 
% $(r_1,r_2,\dots,r_n)$ be selected at random, \emph{independently and uniformly} from $S$.
% Then 
% $$ \Pr{[p(r_1,r_2,\dots,r_n)=0]}\le \frac{d}{|S|}$$
% \end{lemma}

% Unfortunately, this doesn't work over quotient polynomial rings (e.g., instead of
%  $\F[x]$, we have $\F[x]/(x^n+1)$).

% We'll start learning Algebraic structures in hopes we gain enough insight regarding \ref{schawtz-zippel}.
\chapter{Groups Rings and Fields}

\begin{definition}
    A \emph{group} is a tuple $(G,\cdot,e)$ where $G$ is a set and $\cdot$ is a binary operation on $G$
    that satisfies the following constraints:
    \begin{itemize}
        \item the operation holds closure: if $x,y\in G$ so is $x\cdot y\in G$.
        \item the operation is associative: $\forall x,y,z\in G$ it holds that $(x\cdot y)\cdot z = x\cdot (y\cdot z)$.
        \item has an identity element: there exists $e\in G$ such that $\forall x\in G x\cdot e = x = e\cdot x$
        \item Every element $x\in G$ has an inverse $x^{-1}\in G$ such that $x\cdot x^{-1} = e = x^{-1} \cdot x$.
    \end{itemize}
\end{definition}


\begin{remark}
    Note that a group can be an additive group or a multiplicative group.
\end{remark}
\begin{example}
    Assume $\F*=F\\{0}$, $(\F*, \cdot, 1)$ is a valid multiplicative group but,
    $(\F, \cdot, 1)$ is not because there is no inverse to $0$ in $\F$.
\end{example}

\begin{definition}
    group $G$ is called \emph{commutative}, or \emph{abelian} if for every $x,y\in G$ $x\cdot y=y\cdot x$ holds.
\end{definition}


\begin{definition}[monoid] \label{def:monoid}
    A \emph{monoid} is a group without the requirement of inverses.
    That is, a monoid is a tuple $(M,\cdot,e)$ where $M$ is a set and $\cdot$ is a binary operation on $M$
    that satisfies the following constraints:

    \begin{itemize}
    \item the operation holds closure: if $x,y\in G$ so is $x\cdot y\in G$.
    \item the operation is associative: $\forall x,y,z\in G$ it holds that $(x\cdot y)\cdot z = x\cdot (y\cdot z)$.
    \item has an identity element: there exists $e\in G$ such that $\forall x\in G x\cdot e = x = e\cdot x$
    \end{itemize}
\end{definition}


Now that we've defined what a group is, let's highlight some important theorems and definitions:

\begin{definition}
    denote $|G|$ as the \emph{order} of a group - its size.
\end{definition}

\begin{definition}
    denote the order of $x\in G$ to be the minimal $m!=0\in\N$ such that  $x^m=e$. if there is no such $m$, then
    the order of $x$ is infinite.
\end{definition}

\begin{theorem}
    Let there a finite group $G$, thus $\forall x\in G$ has a finite order.
\end{theorem}

\begin{theorem}
    Let $G$ be a group, and let $x\in G$ such that $order(x)=n$, thus $x^m=e$ iff $n|m$.
\end{theorem}

From a group, we can define a field, which is a more complex structure.

\section{Finite Fields}
We are familiar with fields from Linear Algebra and Calculus.
In this section, we introduce finite fields and explore some of their key properties.

\begin{definition}
    A \emph{field} is a tuple $(F,+,\cdot,0,1)$ where $F$ is a set and $+,\cdot$ are binary operations on $F$
    such that $(F,+,0)$ is an abelian group, $(F,\cdot,1)$ is an abelian group, and the distributive law holds.

    \begin{align*}
        \forall x,y,z\in F, && x\cdot(y+z) = x\cdot y + x\cdot z
    \end{align*}
\end{definition}

\begin{example}
    The set of real numbers $\R$ is a field. 
    The set of integers $\Z$ is not a field.
    And every set of integers modulo a prime $p$ is a field.
\end{example}

In the context of finite fields, 
some people use the notation $GF(q)$ to represent a field with a 
size of $q$, while others use $F_q$. In this summary, I'll use them interchangeably.


I'm skipping proving properties over the addition, multiplication, and distributive law,
since they are quite trivial and can be found in any algebra book.

\begin{theorem}
    Let $F$ be a set of $q$ elements, and addition and multiplications are defined 
    over $F$ via modulo $q$. Then $F$ is a field iff $q$ is prime.
\end{theorem}

We usually write the above types of fields as $GF(q)$, $F_q$ or $Z_q$.

\begin{definition}
    Let $F$ be a field. The caracteristic of $F$ is the smallest \bf{positive} integer 
    $p\in F$ such that $p\cdot 1=0$. 
    If no such $p$ exists, the characteristic is $0$.
\end{definition}

For instance, the characteristic of $\Z_p$ is $p$, and the characteristic of $\R$ is $0$.

\begin{theorem}
The characteristic of a field is either $0$ or a prime number.
\end{theorem}

I skip the proofs.

\section{Polynomial Rings}

Now that we defined groups and fields, we can define rings, and then polynomial rings.
\begin{definition}
    A \emph{ring} is a tuple $(R,+,\cdot,0,1)$ where $R$ is a set and $+,\cdot$ are binary operations on $R$
    such that $(R,+,0)$ is an abelian group, $(R,\cdot,1)$ is a monoid, and the distributive law holds.
\end{definition}

In other words, a ring is a group with respect to addition, and a monoid with respect to multiplication (see \autoref{def:monoid}).

\begin{example}
    The set of integers $\Z$ is a ring, since some numbers don't 
    have an inverse (e.g., $2$ doesn't have an inverse in $\Z$). 
\end{example}


\section{Polynomial Rings}

Polynomials are heavily used in cryptography, and are a fundamental concept in algebra.
\begin{definition}
    Let $R$ be a ring, we'll denote $R[X]$ the ring of polynomials
    in $x$ with coefficients in $R$. Formally, an element of $R[X]$ is 
    defined as: $$ \sum_{i=0}^{d} r_i\cdot x^i $$
    where $d$ is the degree of the polynomial, and $r_i\in R$.

    We treat $x$s as symbols, rather than actual values or variables.

    We define polynomial addition and multiplication as follows:
    \begin{align*}
        f(x) + g(x) &= \sum_{i=0}^{d}{(a_i+b_i)x^i} \\
        f(x) \cdot g(x) &= \sum_{i=0}^{d}{\sum_{j=0}^{d}{a_i\cdot b_j\cdot x^{i+j}}}
    \end{align*}
\end{definition} \label{def:polynomial-ring}\label{def:polynomial}

\subsection{Polynomial evaluation}\label{sec:poly-eval}
Since we discuss formal mathematics, it is important to remember 
that while polynomials seem like functions, in our context they are just a formal
symbols.
As a result, to evaluate a polynomial, we'll need to define a function that receives a polynomial 
and a value and returns the result of the polynomial at that value.

\begin{definition} \label{def:poly-eval}
  For any polynomial $P$ in $Z_q[X]$ and any element $a$ in $Z_q$,
  the substitution of $X$ with $a$ in $P$ defines an element of $Z_q$, which is denoted $P(a)$.
  This element is obtained by carrying on in $Z_q$ after the substitution of the operations indicated by 
  the expression of the polynomial.
  We call this action polynomial evaluation of $P$ at $a$.
\end{definition}

Now that we've defined the evaluation of a polynomial,
Let as define a function to evaluate polynomials at a point.

\begin{definition}[polynomial evaluation map]\label{def:poly-eval-map}
  A polynomial evaluation map is a function $f_r:Z_q[X]\to Z_q$ such that for every polynomial $P$ and $r\in Z_q$,
  $f_r(P)=P(r)$.
\end{definition}

This notation of polynomial evaluation maps is crucial to reason about polynomials, their evaluation, 
and proving properties. For instance, assume a matrix $M$ and a vector $v$ with polynomials entries in $\Z_q[x]$.
It isn't clear straight away that $f_r(M\cdot v) = f_r(M)\cdot f_r(v)$. In fact, there are algebraic 
structures where this above equation is false. 
Thus, to claim the above statement we need to define ring homomorphisms, and prove that $f_r$ is a ring homomorphism.

\begin{definition}
  A ring homomorphism is a function $f:R\to S$ between two rings $R$ and $S$ such that for every $a,b\in R$:
  \begin{align}
    &f(a+b)=f(a)+f(b)\\
    &f(a\cdot b)=f(a)\cdot f(b) \\
    &f(1_R)=1_S
  \end{align}
   Where $1_R$ and $1_S$ are the multiplicative identities of $R$ and $S$ respectively.
\end{definition}

Fortunately, for every point in $\Z_q[X]$, we can construct a polynomial evaluation map that is 
a ring homomorphism (prove this to ensure you understand the reasoning behind it).
\begin{theorem}
    Let $f_r:Z_q[X]\to Z_q$ be a polynomial evaluation map, then $f_r$ is a ring homomorphism.
\end{theorem}

Consequently, we can assert that $f_r(M\cdot v) = f_r(M)\cdot f_r(v)$.

\paragraph{GCD and LCM}

Like with other fields and rings, 
we can define modulo operations over polynomials, and as a result
these algebraic strucutres too have GCD and LCM.
\begin{definition}
    Let $f(x)\in F[X]$ be a polynomial of degree $d\ge 1$.
    Then, for any polynomial $g(x)\in F[x]$, there exists a unique 
    pair of polynomials $q(x),r(x)\in F[x]$ such that $g(x)=q(x)\cdot f(x) +r(x)$.
    The polynomial $r(x)$ is called the remainder of $g(x)$ modulo $f(x)$.
\end{definition}

\begin{definition}
    GCD of two polynomials $f(x)$ and $g(x)$ is the monomic polynomial $h(x)$
    of the highest degree which is a divisor of both.
    In particular, we say that $f(x)$ and $g(x)$ are coprime if their GCD is $1$.


    The LCM of these two polynomials is monic polynomial of the lowest 
    degree which is a multiple of both
\end{definition}


\subsection{Polynomial Fields}
\begin{definition}
    Ideal $I$ is a subset of a ring $R$ that is closed under addition and, or multiplication by elements of $R$.

    Furthermore the product of an element of $R$ and an element of $I$ is in $I$.
    That is, $\forall r\in R$ and $i\in I$ it holds that $r\cdot i\in I$.
\end{definition}

For instance, the group $7\Z=\{7\cdot i:i\in\Z\}$ is an ideal of $\Z$ because it is closed
under addition and multiplication by elements of $\Z$,
and the product of an element of $\Z$ and an element of $7\Z$ is also in $7\Z$.

Given an ideal $I$ and a ring $R$, we can define the quotient ring $R/I$.
This ring is defined as the set of equivalence classes of $R$ under the equivalence relation 
$x\sim y$ iff $x-y\in I$ (that is, there exists $i\in I$ such that $x-y=i$).
This $\sim$ notation can be confusing, so we'll denote the equivalence class of $a$ as $[a]=\{b\in \Z :a\sim b \}$.
Furthermore, we note that $[a]+[b]=[a+b]$ and $[a]\cdot[b]=[a\cdot b]$.

\begin{example}
    $\Z/7\Z$ is the set of equivalence classes of $\Z$ under the equivalence relation
    $x\sim y$ iff $x-y\in 7\Z$. 
    For instance, $[0]=7\Z$, $[1]=1+7\Z$, $[2]=2+7\Z$, and so on.

    One can quickly see that there are only 7 equivalence classes, because $[7]=7+7\Z=0+7\Z$ (we wrap around, 
    and don't care about the actual value of the number, only its residue modulo 7).

    This specific quotient ring is also known as $\Z_7$, which is also a field.
\end{example}


\begin{bclogo}[logo=\bcinfo, couleurBarre=orange, noborder=true, couleur=white]{How I think about ideals}
    This formal definition is quite confusing. I think about these ideals as groups 
    with values we don't care about, and $R/I$ defines $R$ without 
    the values we don't care about.

    So in the case of $Z/7Z$, we don't care about 
    all the multiplications of $7$, only the first 7 of them.


    Another important example if the polynomial quotient ring $R[X]/X^d$. 
This ring is the set of equivalence classes of $R[X]$ under the equivalence relation
$f\sim g$ iff $f-g\in X^d$. Or in other words, all polynomials of degree less than $d$ are equivalent.
\end{bclogo}

\begin{definition}[irreducible polynomials]
    Let $F$ be a field, and $f(x)\in F[X]$ be a polynomial of degree $d\ge 1$.
    We say that $f(x)$ is irreducible over $F$ if it cannot be factored into 
    two polynomials of lower degree.
\end{definition}

\begin{definition}[prime polynomials]
    Let $F$ be a field, and $f(x)\in F[X]$ be a polynomial of degree $d\ge 1$.
    We say that $f(x)$ is prime if for every two polynomials $g(x),h(x)\in F[X]$
    such that $f(x)|g(x)\cdot h(x)$, then $f(x)|g(x)$ or $f(x)|h(x)$. (Stronger)
\end{definition}

\begin{bclogo}[logo=\bcinfo, couleurBarre=orange, noborder=true, couleur=white]{primes, irreducibility and roots}
Note, the irreducible or primality of polynomials has nothing to do with roots 
of the polynomial. For instance, $x+1$ is irreducible over $GF(157)$, but has 
a root of $156$.
\end{bclogo}

\begin{theorem}
    Let $f(x)\in F[X]$ be a polynomial of degree $d\ge 1$.
    Then, $F[x]/f(x)$ is a field iff $f(x)$ is irreducible over $F$.
\end{theorem}


\section{Structure of Finite Fields}




\begin{theorem}[Fermat's little theorem]\label{fermats-little-theorem}
    For every element of a finite field $F$ with $q$ elements, 
    $x^q=x$. 
\end{theorem}

\begin{corollary}
    Let $F$ be a finite field with $q$ elements, and let $x\ne0\in F$.
    Then, $x^{q-1}=1$.
\end{corollary}

This theorem is very useful in cryptography, algorithms, and many other fields.
I specifically used it to explain to myself some aspects of the NTT algorithm (see \autoref{section:fft}). 


\begin{corollary}
    Let $F$ be a subfield of $E$ with $|F|=q$. Then 
    an element $x\in E$ is in $F$ iff $x^q=x$.
\end{corollary}



\begin{theorem}
    For any prime $p$ and any integer $n$, there exists a field of size $p^n$.
\end{theorem}

(This is without proof). 
To create such a field:
One can take a field $GF(q)$, and extend it.
To do so, first find a polynomial $p(x)$ of degree $d$ that is irreducible over $GF(q)$
(We'll use prime polynomials).
Then, define the extension field $GF(q^d)$ as the set of polynomials of degree less than $d$ with coefficients in $GF(q)$.
THis field is basically $GF(q)[X]/p(X)$. BUT, Once we have this field, 
we can define it in order, and change the notations of the polynomials to be integers 
in the range $[0,q^d-1]$. This is a way to extend the field, and get a larger field.


For instance $GF(2)$ can be extended to any power $m$ of 2.

For instance, $GF(2)[X]/(X^2+X+1)$ is $GF(2^2)$, and has the following elements:
$0,1,X,X+1$. Once we define the elements and how they interact with their operations, we can 
switch their notation to be integers in the range $[0,3]$. 

\begin{remark}
    Usually,  if we take a root of an irreducible
    polynomial $f(x)$ over $F[X]$, say $\alpha$, we can 
    represent the field $F[X]/f(x)$ as $F[\alpha]$, as follows
    
    $$F[x]/f(x)=F[\alpha]= \{a0 + a1 \alpha + \dots + a_{n-1} \alpha_{d-1} : a_i \in F\}$$
    (we treat $\alpha$ as a symbol, rather than a value). 
\end{remark}


\begin{definition}(generator/primitive element)
    An element $\alpha$ in a finite field $F_q$ is called a primitive element
    (or generator) of $F_q$ if 
    $F_q = \{0, \alpha, \alpha^2, \dots , \alpha^{q-1}\}$.
\end{definition}
That is, a primitive element is a generator 
of the multiplicative group of the field.

\begin{definition}
    The order of a non zero element $\alpha$ in a finite field $F_q$ 
    is the smallest positive integer $m$ such that
    $\alpha^m=1$.
\end{definition}


It is important to note that, if $\alpha$ is a 
root of an irreducible polynomial $f(x)$ over $F[X]$ of degree $m$, 
then it is also a primitive element of $F_{q^m}=F_q[\alpha]$.
As a result, every element can be written as a power of $\alpha$: 
$F_{q^m}=\{0,1,\alpha,\alpha^2,\dots,\alpha^{q^{m-1}}\}$.

Addition, and multiplication is easier while represented in terms of $\alpha$ 
(since it doesn't demand polynomial multiplication and modulo).

\begin{bclogo}[logo=\bcinfo, couleurBarre=orange, noborder=true, couleur=white]{Ordering of elements}
    You don't know the ordering of the elements using the generator $\alpha$.
\end{bclogo}

\paragraph{Zech logarithms}
Zech logarithms are used to implement addition in finite fields 
(via a look up table)
when elements are represented as powers of a generator. 

For sufficiently small finite fields,
 a table of Zech logarithms allows an especially efficient 
 implementation of all finite field arithmetic in terms of a small 
 number of integer addition/subtractions and table look-ups.

The utility of this method diminishes for large fields 
where one cannot efficiently store the table. 
This method is also inefficient when doing very few 
operations in the finite field, because one spends more time 
computing the table than one does in actual calculation.

\section{Tricks}
\todo{If you have any tricks or theorems you think are important, put them here.}
% TODO: discuss Ittai's trick for $A\cap B$ probabiltiy.